 %!TEX TS-program = xelatex
%!TEX encoding = UTF-8 Unicode

%\def \papersize {a5paper}
\def \papersize {a4paper}
%\def \papersize {letterpaper}

%\documentclass[14pt,\papersize]{extarticle}
\documentclass[12pt,\papersize]{extarticle}
% extarticle is like article but can handle 8pt, 9pt, 10pt, 11pt, 12pt, 14pt, 17pt, and 20pt text

\def \ititle {Origins of Mind: Lecture Notes}
\def \isubtitle {Lecture 02}
%comment some of the following out depending on whether anonymous
\def \iauthor {Stephen A.\ Butterfill}
\def \iemail{s.butterfill@warwick.ac.uk% \& corrado.sinigaglia@unimi.it
}
%\def \iauthor {}
%\def \iemail{}
%\date{}

%\input{$HOME/Documents/submissions/preamble_steve_paper3}
\input{$HOME/Documents/submissions/preamble_steve_report2}

%no indent, space between paragraphs
\usepackage{parskip}

%comment these out if not anonymous:
%\author{}
%\date{}

%for e reader version: small margins
% (remove all for paper!)
%\geometry{headsep=2em} %keep running header away from text
%\geometry{footskip=1.5cm} %keep page numbers away from text
%\geometry{top=1cm} %increase to 3.5 if use header
%\geometry{bottom=2cm} %increase to 3.5 if use header
%\geometry{left=1cm} %increase to 3.5 if use header
%\geometry{right=1cm} %increase to 3.5 if use header

% disables chapter, section and subsection numbering
\setcounter{secnumdepth}{-1} 

%avoid overhang
\tolerance=5000

%\setromanfont[Mapping=tex-text]{Sabon LT Std} 


%for putting citations into main text (for reading):
% use bibentry command
% nb this doesn’t work with mynewapa style; use apalike for \bibliographystyle
% nb2: use \nobibliography to introduce the readings 
\usepackage{bibentry}

%screws up word count for some reason:
%\bibliographystyle{$HOME/Documents/submissions/mynewapa} 
\bibliographystyle{apalike} 


\begin{document}



\setlength\footnotesep{1em}






%--------------- 
%--- start paste



\title {Origins of Mind: Lecture Notes \\ Lecture 07}
 
\maketitle
 
 
\subsection{title-slide}
 
\section{Core Knowledge}
 
 
\subsection{slide-3}
In the last lecture, we considered communication with language.
My overall plan now is to work backwards: the next item to consider is communicative action generally, and then its action generally.
I suggested that acquiring an ability to commmunicate with language typically involves (a) social interaction and (b) creating words.
I think looking at communication generally will help bring this into clearer focus.
But before I get into this, I want to take a huge detour.
The huge detour will allow me (i) to connect up the different things we've done; and (ii) pick up on some themes from your assessed essays.
(The detour means that this lecture, unlike the others, doesn't have a single unifying theme.)
 
 
\subsection{slide-4}
I wanted to try to start drawing things together before the end of the last lecture.
This will mean a bit of repetition, but it will also help us in thinking through the issues together.
 
 
\subsection{slide-5}
This isn't a new question, but I think it's worth spending more time on it, partly because someone said in a seminar that we don't know what it is and partly because thinking about this is a way of organising much of what we've been learning.
 
 
\subsection{slide-7}
The first, very minor thing is to realise that there are two closely related notions, core knowledge and core system.
The notion of a core system and that of a module are barely different; it's safe to treat these as interchangeable until you have a reason to distinguish them.
These are related this: roughly, core knowledge states are the states of core systems. More carefully:
For someone to have \textit{core knowledge of a particular principle or fact} is for her to have a core system where either the core system includes a representation of that principle or else the principle plays a special role in describing the core system.
So we can define core knowlegde in terms of core system.
 
 
\subsection{slide-8}
The next step is to realise that there is a good reason why you don't know what core systems are.
 
 
\subsection{slide-9}
Core systems are usually introduced implicitly, in explaining an idea.
So to properly understand what they are we would need (i) to have a deep understanding of the picture; (ii) and of the hypotheses it inspires; (iii) and of the evidence for these hypotheses, and then we would work back from this to say what core systems are.
Now you might say that this is terrible, how can scientists use terms without defining them.
But (a) it's not obvious that definitions are necessary for good science, or even that useful; and (b) compare the notion of knowledge: philosophers have made some informative observations about knowledge, but they've had no success at all in defining it.
(That said, I do think there's a particular problem with core knowledge.)
 
 
\subsection{slide-10}
In answering the question, What is core knowledge? I think we should be inspired by the notion of radical interpretation.
‘All understanding of the speech of another involves radical interpretation’
\citep[p.\ 125]{Davidson:1973jx}
(It's not just core knowledge; I think we have too approach science as radical interpreters ...)
How does radical interpretation work?
Interpretation is hard because there are two factors: truth and meaning.
The proposal Davidson makes is that we assume truth and infer meaning.
I'm recommending a similar strategy for core knowledge.
Very roughly, we take for granted that the evidence establishes various hypotheses. We then ask what core knowledge could be given these are true.
But more carefully, we first have to ask what motivates talk about core knowledge at all.
 
 
\subsection{slide-11}
Fine, but this doesn't help us in practical terms. How are we to get a handle on the notion without doing lots of research?
The simple approach is to find out what people who use the term say it is.
 
 
\subsection{slide-12}
What do people say core knowledge is?
There are two parts to a good definition. The first is an analogy that helps us get a fix on what we is meant by 'system' generally. (The second part tells us which systems are core systems by listing their characteristic features.)
 
 
\subsection{slide-14}
So talk of core knowledge is somehow supposed to latch onto the idea of a system.
What do these authors mean by talking about 'specialized perceptual systems'?
They talk about things like perceiving colour, depth or melodies.
Now, as we saw when talking about categorical perception of colour, we can think of the 'system' underlying categorical perception as largely separate from other cognitive systems--- we saw that they could be knocked out by verbal interference, for example.
So the idea is that core knowledge somehow involves a system that is separable from other cognitive mechanisms.
As Carey rather grandly puts it, understanding core knowledge will involve understanding something about 'the architecture of the mind'.
 
 
\subsection{slide-15}
Illustration: edge detection.
 
 
\subsection{slide-18}
This is helpful for getting started.
But we quickly run into the problem that different researchers say different things, and it isn't obvious which differences matter.
We also run into the problem that the definitions on offer aren't obviously correct: they list features that maybe aren't all necessary.
 
 
\subsection{slide-19}
Aside: compare the notion of a core system with the notion of a module
The two definitions are different, but the differences are subtle enough that we don't want both.
My recommendation: if you want a better definition of core system, adopt core system = module as a working assumption and then look to research on modularity because there's more of it.
An example contrasting Grice and Davidson on the wave.
 
 
\subsection{slide-27}
So now we have a rough, starting fix on the notion we can ask a deeper question.
 
 
\subsection{slide-28}
So why do we need a notion like core knowledge?
Think about these domains.
In each case, we're pushed towards postulating that infants know things, but also pushed against this.
Resolving the apparent contradiction is what core knowledge is for.
Key question: What features do we have to assign to core knowledge if it's to describe these discrepancies?
I think the fundamental feature is inaccessibility.
 
 
\subsection{slide-34}
If this is what core knowledge is for, what features must core knowledge have?
 
 
\subsection{slide-35}
To say that a system or module exhibits limited accessibility is to say that the representations in the system are not usually inferentially integrated with knowledge.
I think this is the key feature we need to assign to core knowledge in order to explain the apparent discrepancies in the findings about when knowledge emerges in development.
 
 
\subsection{slide-36}
Limited accessbility is a familar feature of many cognitive systems.
When you grasp an object with a precision grip, it turns out that there is a very reliable pattern.
At a certain point in moving towards it your fingers will reach a maximum grip aperture which is normally a certain amount wider than the object to be grasped, and then start to close.
Now there's no physiological reason why grasping should work like this, rather than grip hand closing only once you contact the object.
Maximum grip aperture shows anticipation of the object: the mechanism responsible for guiding your action does so by representing various things including some features of the object.
But we ordinarily have no idea about this.
The discovery of how grasping is controlled depended on high speed photography.
This is an illustration of limited accessibility.
(This can also illustrate information encapsulation and domain specificity.)
 
 
\subsection{slide-37}
This picture is significantly different from some competitors (but not Carey on number):
(1) because it shows we aren't done when we've explained the acquisition of core knowledge (contra e.g. Leslie, Baillargeon), and
(2) because it shows we can't hope to explain the acquisition of knowledge if we ignore core knowledge (contra e.g. Tomasello)
 
 
\subsection{slide-38}
***todo*** say something about what we've learnt from each case.
Syntax is important because it pushes us away from the idea of 'systems that humans share with animals' \citep[p.\ 2784]{spelke:2012_core}
Or maybe identify themes and point out which cases they're
e.g. colour shows (i) that perceptual mechanisms are important and (ii) that infants' core knowledge persists into adulthood
 
 
\subsection{slide-39}
Let me pause to evalaute the picture I offered in lecture 1 in the light of what we've learnt so far.
(I hesitate to do this because it shows the picture I offered you isn't very good.)
Take each case in turn.
For colour it works quite well, providing, as I suggested last week, that acquiring words is a creative process of social interaction.
What about physical objects? Here there's no indication that using labels for objects drives a developmental change, and it's hard to see why it would.
(It's more plausible that tool use rather than word use matters; but even this is is hugely speculative.)
So no marks for that case at all.
What about minds, and in particular beliefs?
Superficially things look better here. There is both evidence that rich forms of social interaction facilitate development \citep{Hughes:2006fu}
and also evidence that language matters in various ways \citep{Astington2005ot}.
But these probably don't connect in the simple way I envisage.
Social interaction might matter because it provides experiences of perspective differences, or because it motivates children to think about others' minds.
And language might matter because having sentences around enables them to keep track of beliefs, or because using relative clauses might clue them in to a relation between beliefs and what utterances of sentences express.
So here the picture isn't right, but it might not be a million miles off either ...
It's wrong to think that labelling beliefs matters; but it may be that being able to talk about beliefs (implicitly or otherwise) does matter for coming to have knowledge of them.
 
 
\subsection{slide-40}
So far we've ignored what is usually regarded as a paradigm case for core knowledge ...
 
 
\subsection{unit\_641}
 
\section{Syntax / Innateness}
 
 
\subsection{slide-42}
So far we have considered examples of core knowledge. But we have ignored a paradigm case, one which has inspired much work on this topic ...
 
 
\subsection{slide-43}
Human adults have extensive knowledge of the syntax of their languages, as illustrated by, for example, their abilities to detect grammatical and ungramatical sentences which they have never heard before, independently of their meanings. To adapt a famous example from Chomsky, ...
 
 
\subsection{slide-44}
We need to task two questions.
 
 
\subsection{slide-47}
First, what is this thing, syntax, which is known?
This thing they know, the syntax, isn't plausibly just a list of which sentences are grammatical.
Because people can make judgements about arbitrarily long, entirely novel sentences.
Rather, the thing known must be something that enables people to make judgements about sentences.
We might think of it roughly as a theory of syntax.
It's like a theory in this sense: knowledge of it enables you to make judgements about the grammaticality of arbitrary sentences.
 
 
\subsection{slide-52}
The second question is, Is it *knowledge* we have syntax or something else?
There's something interesting.
The knowledge can be revealed indirectly, by asking people about whether particular sentences are grammatical.
But people can't say anything about how they know the sentence is grammatical.
It's like perceiving the shape of something: there isn't much to say about how you know.
So the theory of syntax isn't something we can discover by introspection:
we have to *rediscover* it from scratch by investigating people's linguistic abilities.
 
 
\subsection{slide-56}
Knowledge of syntax therefore seems to have some of the features associated with core knowledge.
First, it is domain-specific.
Second, it is inaccessible. That is, it can't guide arbitrary actions.
In what follows I want to suggest that syntax provides a paradigm case for thinking about core knowledge.
 
 
\subsection{slide-62}
In addition, I want to use the case of syntax for thinking about the question, What is innate in humans?
I was astonished how many people considered this question in the unassessed essay, some people seem really fascinated by it.
But almost no one discussed the case of syntax in depth. If you're going to talk about innateness, you really need to know a little bit about syntax.
So I'm also going to provide you with that understanding.
 
 
\subsection{slide-63}
Consider a phrase like 'the red ball'.
What is the syntactic structure of this noun phrase?
In principle there are two possibilities.
 
 
\subsection{slide-68}
How can we decide between these?
\begin{enumerate} \item ‘red ball’ is a constituent on (b) but not on (a) \item anaphoric pronouns can only refer to constituents \item In the sentence ‘I’ll play with this red ball and you can play with that one.’, the word ‘one’ is an anaphoric prononun that refers to ‘red ball’ (not just ball). \citep{lidz:2003_what,lidz:2004_reaffirming}.
\end{enumerate}
 
 
\subsection{slide-71}
What I've just shown you is, in effect, how we can decide whihc way an adult human understands a phrase like 'the red ball'.
We can discover this by finding out how they understand a sentence like 'I’ll play with this red ball and you can play with that one.'.
But how could we do this with infants who are incapable of discussing sentences with us?
 
 
\subsection{slide-72}
Here's how the experiment works (see \citealp{lidz:2003_what}) ...
The experiment starts with a background assumption:
‘The assumption in the preferential looking task is that infants prefer to look at an image that matches the linguistic stimulus, if one is available’ \citep{lidz:2003_what}.
 
 
\subsection{slide-74}
So the key question was whether infants would look more at the yellow bottle (which is familiar) or the blue bottle (which is novel).
If they think 'one' refers to 'bottle', we'd expect them to look longer at the blue bottle;
and conversely if they think one refers to 'yellow bottle', then they're being asked whether they see another yellow bottle.
 
 
\subsection{slide-76}
And, as always, we need a control condition to check that infants aren't looking in the ways predicted irrespective of the manipulation.
 
 
\subsection{slide-77}
And here's what they found ...
 
 
\subsection{slide-78}
What can we conclude so far?
So there is core knowledge of syntax ... or is there?
 
 
\subsection{slide-79}
Core knowledge is often characterised as innate.
I think this is a mistake (more about this later), but many of you do not.
How could we tell whether these representations are innate?
 
 
\subsection{slide-82}
What do we mean by innate here?
The easy answer is: not learned.
But I think there's a more interesting way to approach understanding what 'innate' means.
Quite a few people pointed out that there isn't agreement on what innateness is.
But this is not very interesting by itself because there's disagreement about most things and potential causes of disagreement include ignorance and stupidity.
It's also important that the mere fact that a single term is used with multiple meanings isn't an objection to anyone.
As philosophers, some of you are tempted to catalogue different possible notions of innateness.
I encourage you to resist this temptation; if you want to collect something, pick something useful like banknotes.
There's a much better way to approach things.
Let's see what kind of findings are, or would be, taken to show that something is innate.
We can use these to constrain our thinking about innateness.
We will say: assuming that this is a valid argument that X is innate, what could innateness be?
 
 
\subsection{slide-83}
Aside: we have too approach science as radical interpreters ...
How does radical interpretation work?
Interpretation is hard because there are two factors: truth and meaning.
The proposal Davidson makes is that we assume truth and infer meaning.
I'm recommending a similar strategy.
We take for granted that this argument establishes that X is innate; we then ask what innateness could be given that this is so.
‘All understanding of the speech of another involves radical interpretation’
\citep[p.\ 125]{Davidson:1973jx}
 
 
\subsection{slide-84}
The best argument for innateness is the poverty of stimulus argument.
We need to step back and understand how poverty of stimulus arguments work.
Here I'm following \citet{pullum:2002_empirical}, but I'm simplifying their presentation.
How do poverty of stimulus arguments work? See \citet{pullum:2002_empirical}.
First think of them in schematic terms ...
This is a good structure; you can use it in all sorts of cases, including the one about chicks' object permanence.
Now fill in the details ...
 
 
\subsection{slide-92}
In our case, X is knowledge of the syntactic structure of noun phrases. (Caution: this is a simplification; see\citet[p,\ 158]{lidz:2004_reaffirming}).)
 
 
\subsection{slide-94}
This is what the Lidz et al experiment showed.
Note that no one takes this to be evidence for innateness by itself.
 
 
\subsection{slide-96}
What is the crucial evidence infants would need to learn the syntactic structure of noun phrases?
This is actually really hard to determine, and an on-going source of debate I think.
But roughly speaking it's utterances where the structure matters for the meaning, utterances like 'You play with this red ball and I'll play with that one'.
 
 
\subsection{slide-98}
\citet{lidz:2003_what} establish this by analysing a large corpus (collection) of conversation involving infants.
 
 
\subsection{slide-101}
What can we infer about innateness from this argument?
First, think about what is innate. The fact that knowledge of X is acquired other than by data-driven learning doesn't mean that X is not innate; it just means that something which enables you to learn this is.
Second, think about the function assigned to innateness. That which is innate is supposed to stand in for having the crucial evidence.
This, I think, is the key to thinking about what we *ought* to mean by innateness.
So attributes like being genetically specified are extraneous---they may be typical features of innate things, but they aren't central to the notion.
By contrast, that what is innate is not learned must be constitutive (otherwise that which is innate couldn't stand in for having the crucial evidence)
 
 
\subsection{slide-102}
Contrary to what many philosophers (including Stich and Fodor) will tell you ...
 
 
\subsection{slide-103}
I asked you this question, but what do I think?
I'd approach it by distinguishing two sub-questions (the second of which has two sub-sub-questions)
 
 
\subsection{slide-107}
Arguments from the poverty of stimulus are the best way to establish innateness.
The argument concerning syntax we've just been discussing is quite convincing, although if you follow up on the references given in the handout you'll see it's not decisive (as always).
For things other than knowlegde of syntax, the evidence concerning humans is far less clear.
There are, however, quite good cases in nonhuman animals, as many of you know.
So it's not unreasonable to conjecture that learning in the several domains where infants appear to know things early in their first year is innately-primed rather than entirely data-driven.
But, one or two cases aside, there's enough evidence to rule out the converse conjecture.
 
 
\subsection{slide-110}
I don't think what is innate is knowledge, nor do I think it's concepts.
But I think there's a good chance that modules are innate (and therefore core knowledge if I'm right to suppose that 'core knowledge' is a term for the fundamental principles describing the operation of a module).
 
 
\subsection{slide-113}
On content: I think quite a lot is known about the modules thanks to detailed tests that have little to do directly with controversy about inateness.
 
 
\subsection{slide-114}
Why care about whether something is innate? (This isn't suppose to be dismissive.)
 
 
\subsection{slide-115}
Here are two reasons why I think we shouldn't worry too much about innateness in trying to understand the origins of mind.
(1) The question about innateness concerns the first transition, whereas I think the second should be our focus (for pragmatic reasons: there's more research).
(2) Discoveries about innately-primed learning make only a relatively modest contribution to understanding the emergence of core knowledge in development. So even when we consider the first transition, it's not obvious that discoveries about innateness are very illuminating, for all their pop-science appeal.
Metaphor: we find a cake in the ruins of Pompeii preserved for a couple of thousand years. We're trying to reconstruct its manufacture.
Its good if someone obsesses about where the eggs came from. Did the baker have her own chickens or did she get them from a friend?
But knowing where the eggs came from is unlikely to be critical to understanding how the cake was manufactured.
We're not finished when we know where the eggs came from, and we're not doomed to fail if we don't know.
 
 
\subsection{slide-116}
So let me put the innateness issue aside and get back to what I think matters most ...
 
 
\subsection{slide-118}
This paradigm allows me to highlight something about core knowledge.
I would be a mistake to suppose that there is some core knowledge which later becomes knowledge proper --- e.g. the fact that barriers stop solid objects is first core knowledge then later knowledge.
The content of the core knowledge is a theory of syntax (let's say).
Or, in another case, the content of core knowledge is some principles of object perception.
These are things that human adults do not typically know at all, at least not in the sense that they could state the principles.
So core knowledge enables us to do things, like anticipate where unseen objects will re-appear or communicate with words.
It doesn't seem to be linked directly to the acquisition of concepts.
 
 
\subsection{slide-120}
Now I want to go back the main line of my plan, which calls for us to consider communication generally.
 
 
\subsection{unit\_661}
 
\section{Pointing}
 
 
\subsection{slide-122}
Tomasello calls the third kind 'declarative'.
 
 
\subsection{slide-123}
Infants spontaneously point from around 12 months.
Here is a situation in which a child is playing at her table. Then something appears from behind a sheet. The infant spontaneously points at it.
 
 
\subsection{slide-124}
Infants point to intiate joint engagement Liszkowski et al 2006.
‘Four hypotheses about what infants want when they point were tested. First, on the hypothesis that infants pointed for themselves (see above), E neither attended to the infant nor to the event (Ignore condition). Second, on Moore and D’Entremont’s (2001) hypothesis that infants do not want to direct attention and just want to obtain attention to themselves, E never looked at the event and instead attended to the infant’s face and emoted positively to it (Face condition). Third, on the hypothesis that infants just wanted to direct attention and nothing else, E only attended to the events (Event condition). Fourth, on our hypothesis that infants want to share attention and interest, E responded to an infant’s point by alternating gaze between the event and the infant, emoting positively about it (Joint Attention condition).’ \citep{Liszkowski:2007mm}
‘When interacting with an adult who always reacted consistently in one of four ways, 12-month-olds pointed most often across trials if the adult actively shared her attention and interest in the event (Joint Attention condition)’ \citep[p.\ 305]{liszkowski:2004_twelve}
‘Analyses of infants’ points within each event revealed a complementary set of results. In the conditions not involving joint attention, infants repeated their point more often. This repeating behavior presumably indic- ates that they were dissatisfied with the adult’s response, and so they were persisting in their pointing behavior hoping eventually to obtain the desired response (which was presumably joint attention, since children did not repeat themselves very often in this condition).’ \citep{Liszkowski:2007mm}
 
 
\subsection{slide-126}
Now imagine an experiment with four conditions.
In each condition, there are several trials involving something appearing and, hopefully, the infant pointing at it.
So how do the conditions differ?
 
 
\subsection{slide-128}
In one condition, the experimenter ignores the infant when she points.
 
 
\subsection{slide-131}
In another condition, the experimenter looks at the infant only.
 
 
\subsection{slide-139}
What predictions should we make?
 
 
\subsection{slide-141}
If infants point to draw attention to themselves, what can we predict?
 
 
\subsection{slide-143}
They should be more satisfied in these conditions.
 
 
\subsection{slide-145}
They should be less satisfied in these conditions.
But how can we measure satisfaction?
Within a trial: less satisfied with response should lead to more pointing.
Across all trials: more satisfied with responses should make it more likely that pointing will occur in a trial (at least once).
 
 
\subsection{slide-150}
If infants point to initiate joint engagement, what should we expect then?
 
 
\subsection{slide-152}
Satisfied in this condition and not in any other.
 
 
\subsection{slide-153}
So here's the setup again (but schematically this time).
 
 
\subsection{slide-154}
And here is the first key finding: more pointing overall (across trials) when there's joint attention.
[*todo: redraw Ulf's figures and put this \& next on one slide]
 
 
\subsection{slide-155}
And here is the second key finding: less pointing within a trials when there's joint attention.
 
 
\subsection{slide-158}
Why is this significant?
Because it implies two things:
First, it implies that infants' pointing is referential communication; that is, communication about an object.
(Contrast sharing a smile; we're communicating, but not necessarily referring.)
Second, it implies that infants have some understanding of joint engagement.
I'll come back to this in later.
 
 
\subsection{slide-159}
By the way, this is related to the essay topic on shared intentionality.
Tomasello takes infants' pointing to be based on what he calls shared intentionality.
I don't want to pursue this here; I'm just mentioning it because you might want to draw on pointing in your seminar essays.
Instead I want to ask in a simpler way, how should we understand pointing as other non-linguistic forms of communication?
To approach this question, it's helpful to compare and contrast humans with other, nonhuman apes.
 
 
\subsection{slide-166}
What does declaratively mean? Liszkowski and Tomasello call pointing declarative when its done to initiate joint engagement.
 
 
\subsection{slide-170}
So the discrepancy is not easily explained.
Comprehension is also missing ...
 
 
\subsection{slide-172}
In this experiment, we contrast failed reaches with pointing ...
Hare and Call (\citeyear{hare_chimpanzees_2004}) contrast pointing with a failed reach as two ways of indicating which of two closed containers a reward is in. Chimps can easily interpret a failed reach but are stumped by the point to a closed container.
You are the subjects. This is what you saw (two conditions). Your task was to choose the container with the reward.
Infants can do this sort of task, it's really easy for them \citep{Behne:2005qh}. (And, incidentally, they distinguish communicative points from similar but non-communicative bodily configurations.)
The pictures in the figure stand for what participants, who were chimpanzees, saw.
The question was whether participants would be able to work out which of two containers concealed a reward.
In the condition depicted in the left panel, participants saw a chimpanzee trying but failing to reach for the correct container.
Participants had no problem getting the reward in this case, suggesting that they understood the goal of the failed reach.
In the condition depicted in the right panel, a human pointed at the correct container.
Participants did not get the reward in this case as often as in the failed reach case, suggesting that they failed to understand the goal of the pointing action.
(Actually the apes were above chance in using the point, just better in the failed reach condition. Hare et al comment ;chimpanzees can learn to exploit a pointing cue with some experience, as established by previous research (Povinelli et al. 1997; Call et al. 1998, 2000), and so by the time they engaged in this condition they had learned to use arm extension as a discriminative cue to the food’s location' \citep[p.\ 578]{hare_chimpanzees_2004}.)
\footnote{ The contrast between the two conditions is not due merely to the fact that one involves a human and the other a chimpanzee. Participants were also successful when the failed reach was executed by a human rather than another chimpanzee \citep[][experiment 1]{hare_chimpanzees_2004}. }
\textbf{Note that} chimpanzees do follow the point to a container \citep[see][p.\ 6]{Moll:2007gu}.
 
 
\subsection{slide-177}
Tomasello also asks this question.
'the specific behavioral form---distinctive hand shape with extended index finger---actually emerges reliably in infants as young as 3 months of age (Hannan \& Fogel, 1987). [...] why do infants not learn to use the extended index finger for these social functions at 3 – 6 months of age, but only at 12 months of age?' \citep[p.\ 716]{Tomasello:2007fi}
Again his answer involves shared intentionality: it's because they don't understand shared intentionality until around their first birthdays.
 
 
\subsection{slide-178}
I want to say a tiny bit more on what is involved in understanding a pointing gesture.
Suppose that we are doing puzzle. Then if I point to a piece, I probably intend you to do something with it in the context of our activity.
By contrast, if we are tidying up, a point to the same object might mean something different.
So:
Comprehending pointing is not just a matter of locking onto the thing pointed to; it also involves some sensitivity to context \citep[see][]{Liebal:2010lr}.
This is nicely brought out in a study by Christina Liebel and others.
 
 
\subsection{slide-182}
18-month-olds can do this, but 14-month-olds can't. (Don't infer anything from null result.)
 
 
\subsection{slide-184}
‘Already by age 14 months, then, infants interpret communication cooperatively, from a shared rather than an egocentric perspective’ \citep[p.\ 269]{Liebal:2010lr}.
‘The fact that infants rely on shared experience even to interpret others’ nonverbal pointing gestures suggests that this ability is not specific to language but rather reflects a more general social-cognitive, pragmatic understanding of human cooperative communication’ \citep[p.\ 270]{Liebal:2010lr}.
 
 
\subsection{slide-185}
This makes sense of why chimpanzees don't point --- they don't understand communicative intention.
Compare \citep[p.\ 516]{Tomasello:2010dy}: ‘they do not understand communicative intentions’
 
 
\subsection{slide-186}
But is this consistent with the findings that 12-month-old infants do point?
What is it to understand that by this pointing action another intends to communicate that?
 
 
\subsection{unit\_671}
 
\section{What is a communicative action?}
 
 
\subsection{slide-189}
What is a communicative action?
Why are we asking this question?
Let me try to explain it like this ...
 
 
\subsection{slide-190}
Why do we want to consider what communicative actions are?
Here's what I take to be the view of Tomasello and colleagues.
 
 
\subsection{slide-193}
No, I don't know what that is either. I've asked you to find out in the last essay for this course.
 
 
\subsection{slide-197}
If the theory of communicative action is correct, then the claims about development are incompatible.
 
 
\subsection{slide-198}
So, apparently, if Tomasello et al are right, I'm wrong about two things:
First, I'm wrong that knowlegde of others' minds first emerges around three or four years of age and that one-year-old infants have only core knowledge of mental states.
And, seccond, I'm probably also wrong to think that abilities to communicate (whether by language or not) could explain the emergence of knowledge of others' minds.
This is one reason for asking, What is a communicative action?
 
 
\subsection{slide-199}
Let’s start with some simple examples of non-communicative actions.
Purely physical interaction.
 
 
\subsection{slide-202}
Why are these actions non-communicative? Ayesha intends her fake yawn to have an effect on Ben, but the effect is a physiological one. The response she wants from him is mechanical.
But there are also non-communicative actions which require a rational response from the people they’re directed at. For example:
 
 
\subsection{slide-205}
Note here that although Ayesha intends to provide Ben with misinformation, her action isn’t communicative.
Intuitively, there’s a difference between deliberately providing information or misinformation to someone and communicating with her (Grice 1989: 218).
So what makes an action communicative?
Paul Grice has a neat answer to this question.
He notes that we sometimes achieve things merely by letting other people know that we intend to achieve them.
Waving is one of the simplest examples:
 
 
\subsection{slide-208}
In this example, Ayesha’s goal is to get Ben to come over. Her means of achieving this is to get Ben to recognise that this is what she intends. So when she waves, her intention is that waving will let Ben know that she intends him to come over.
 
 
\subsection{slide-209}
You can achieve some things just by letting people know that you intend to achieve them. To achieve things in this way is to perform an act of communication.
 
 
\subsection{slide-211}
Note that, on this Gricean view, communicating involves having intentions about intentions.m
 
 
\subsection{slide-212}
First approximation: To communicate is to provide someone with evidence of an intention with the further intention of thereby fulfilling that intention
To communicate, then, is to attempt to fulfil an intention by making it manifest to someone else that you have this intention. If you’ve studied Grice you’ll know that his analysis of meaning led to a long and boring series of counterexamples and refinements, most of which shed no light on the nature of linguistic communication (Schiffer 1987). But what’s really important about Grice isn’t the attempt to analyse meaning: it’s his insight
 
 
\subsection{slide-213}
What are the alternatives?
I want to mention two alternatives ...
 
 
\subsection{slide-215}
One alternative is inspired by opponents of the claim, inspired by Grice, that communication by language involves identifying utterer's intentions.
Inspired by Grice, you might think that this is fundamental to linguistic communication.
But philosophers like Dummett and (for different reasons) Millikan reject this view.
We might try to provide an account of pointing in which it's not fundamentally a matter of intention at all.
This would be a radical departure from the Gricean view about pointing. But there is another alternative, one which is less radical.
 
 
\subsection{slide-217}
Like Grice:
 
 
\subsection{slide-218}
We need to distinguish ulterior intentions from semantic intentions.
ulterior intentions: ‘intentions which lie as it were beyond the production of words … [such as] the intention of being elected mayor, of amusing a child, of warning a pilot of ice on the wings’ \citep[p.\ 298]{Davidson:1992pl}.
semantic intentions: intentions concerning the meaning of one’s utterance.
Why does this distinction matter?
Grice’s explicates meaning and communication in terms of ulterior intentions.
His project is to give a reductive analysis of these notions, meaning and communication.
Ulterior intentions are precisely what is needed for such an analysis of meaning.
This is because ulterior intentions ‘do not involve language, in the sense that their description does not have to mention language’ or any semantic concepts like meaning \citep[p.\ 298]{Davidson:1992pl}.
But, Davidson points out, we don’t have to attempt an analysis of meaning and communication.
After all, Grice’s analysis has been subject to plenty of counterexamples and objections \citep{Schiffer:1987zb}.
(Davidson objects that ‘it is not clear that these principles [Grice’s] are designed to handle the gamut of examples we find in literature’ \citep[p.\ 300]{Davidson:1992pl}.
\citet{Davidson:1991ic} discusses one literary example at length. He argues that ‘Joyce takes us back to the foundations and origins of communication; he puts us in the situation of the jungle linguist trying to get the hand of a new language and a novel culture, to assume the perspective of someone who is an alien or an exile’ \citep[p.\ 11]{Davidson:1991ic}.)
 
 
\subsection{slide-219}
This aim ‘assumes the notion of meaning’, but it is important because ‘it provides a purpose which any speaker must have in speaking, and thus constitutes a norm against which speakers and others can measure the success of their verbal behavior.’ \citep[p.\ 11]{Davidson:1994ol}
*todo: this is linked to how Davidson distinguishes first meaning from pragmatic bits; see 'meaning is a psychological concept v2' (for Martin Davies)
But how does this idea translate into a claim about what a pointing action is?
First consider the wave from earlier ...
 
 
\subsection{slide-220}
An example contrasting Grice and Davidson on the wave.
These intentions have a means-end ordering; the ulterior intention is further down the means-end chain.
Strictly speaking, that Ben should come over might not be the first meaning of the wave (so there are other options here).
 
 
\subsection{slide-221}
Now contrast Grice and Davidson on the pointing action from the Hare et al study, where you're supposed to take one of two containers.
Strictly speaking, that Ben should come over might not be the first meaning of the wave (so there are other options here).
 
 
\subsection{slide-222}
The question was, Should we accept that pointing (and linguistic communication) involves intentions about intentions?
(*Davidson himself thinks all communication involves sophisticated insights into others’ minds …)
Is this enough to save the view about development I wanted to offer. Maybe, maybe not …




%--- end paste
%--------------- 
 





\bibliography{$HOME/endnote/phd_biblio}



\end{document}