 %!TEX TS-program = xelatex
%!TEX encoding = UTF-8 Unicode

%\def \papersize {a5paper}
\def \papersize {a4paper}
%\def \papersize {letterpaper}

%\documentclass[14pt,\papersize]{extarticle}
\documentclass[12pt,\papersize]{extarticle}
% extarticle is like article but can handle 8pt, 9pt, 10pt, 11pt, 12pt, 14pt, 17pt, and 20pt text

\def \ititle {Origins of Mind: Lecture Notes}
\def \isubtitle {Lecture 02}
%comment some of the following out depending on whether anonymous
\def \iauthor {Stephen A.\ Butterfill}
\def \iemail{s.butterfill@warwick.ac.uk% \& corrado.sinigaglia@unimi.it
}
%\def \iauthor {}
%\def \iemail{}
%\date{}

%\input{$HOME/Documents/submissions/preamble_steve_paper3}
\input{$HOME/Documents/submissions/preamble_steve_report2}

%no indent, space between paragraphs
\usepackage{parskip}

%comment these out if not anonymous:
%\author{}
%\date{}

%for e reader version: small margins
% (remove all for paper!)
%\geometry{headsep=2em} %keep running header away from text
%\geometry{footskip=1.5cm} %keep page numbers away from text
%\geometry{top=1cm} %increase to 3.5 if use header
%\geometry{bottom=2cm} %increase to 3.5 if use header
%\geometry{left=1cm} %increase to 3.5 if use header
%\geometry{right=1cm} %increase to 3.5 if use header

% disables chapter, section and subsection numbering
\setcounter{secnumdepth}{-1} 

%avoid overhang
\tolerance=5000

%\setromanfont[Mapping=tex-text]{Sabon LT Std} 


%for putting citations into main text (for reading):
% use bibentry command
% nb this doesn’t work with mynewapa style; use apalike for \bibliographystyle
% nb2: use \nobibliography to introduce the readings 
\usepackage{bibentry}

%screws up word count for some reason:
%\bibliographystyle{$HOME/Documents/submissions/mynewapa} 
\bibliographystyle{apalike} 


\begin{document}



\setlength\footnotesep{1em}






%--------------- 
%--- start paste



\title {Origins of Mind: Lecture Notes \\ Lecture 06}
 
\maketitle
 
 
\subsection{title-slide}
 
\section{An utterance}
 
 
\subsection{slide-3}
Linguistic communication is 'the indispensable instrument of fine-grained interpersonal understanding' (Davidson 1990: 326).
It enables us to pool knowledge and coordinate our actions.
Possessing language also allows us to share our lives in ways far more intimate than we could manage without it.
Today's question is, How could humans come to have abilities to communicate by language?
This question is complex because abilities to communicate involve both comprehension and production.
 
 
\subsection{slide-4}
Today's question is also complex because we're starting at the far end of a chain.
You might thing, before we ask about communication by language we should ask about communicative actions, and before we ask about communicative actions we should ask about actions.
I wondered about aproaching it that way too, starting with action and working up.
But in the end I thought that right now you most of you are probably most interested in language, so most of you would probably find it most interesting to go backwards.
 
 
\subsection{slide-5}
Today's question is also complex because we don't actually know very much about the nature of communication by language.
This makes talking about the *acquisition* of abilities to communicate with words tricky.
We lack a clear understanding of the thing to be acquired.
Sperber and Wilson attempt to explain do human beings communicate with one another.
You might say, well let's not study acquisition at all, let's study the endpoint first.
But I don't think this is an option because understanding what the endpoint is and understanding how it is acquired are things we have to do simultaneously.
 
 
\subsection{slide-6}
Chomsky holds that ‘the topic of successful communication in the actual world is far too complex and obscure’ to say anything systematic about (Chomsky 1992: 120). It’s a mistake, he thinks, to try to understand language by focussing on communication (compare Chomsky 2000: 30). I take Chomsky’s position seriously: we are going to have to question whether we’re asking the right question.
If we are to make the topic of communiation with language tractcbale at all, we must start by breaking the question down.
To this end we need to ask ...
 
 
\subsection{slide-7}
As background we have to ask, What is language?
The truth is extremely complex.
But we need only a relatively simple picture ...
 
 
\subsection{slide-8}
Consider an utterance of sentence like 'Isabel slept and Lily cried'.
What is the structure of this utterance?
First we can break the sentence down into two sentences and a connective.
Then we break these into words.
And the words break down into phonemes.
Which are eventually realised in continous bodily movements and the sounds these produce.
This is a very simplified picture.
But already we can see that comprehending an utterance will involve several steps.
Consider someone experiencing linguistic communication for the first time.
She is experiences continuous bodily movements and their acoustic consequences.
From these she has somehow to extract the phonetic gestures.
And then she has to group the phonetic gestures into words.
And finally she has to work out the syntactic structure of the words.
How does this help?
It turns out that the abilitues to make these different steps involves largely different mechanisms, and that the steps can be made independently of each other (not that there aren't bottom-up and top-down effects, just that these appear to be inessential in some cases).
Our question can be broken down accordingly.
Our question How do humans come to have abilities to communicate by language?
We can break the process of language comprehension into a series of transitions, from bodily movements and their acoustic effects to phonemes \&c.
And we can think of language production as involving the same transitions, but in reverse.
And then our question can be broken down accordingly.
We don't have to ask how humans come to have abilities to communicate by language all in one go.
Instead we can ask how humans come to be able to identify phonemes in continous bodily movements and their acoustic effects, and how they come to identify words from uninterrupted sequences of phonemes, and so on.
In this way, our question becomes tractable.
 
 
\subsection{slide-9}
So far I haven't mentioned the one feature of communication by language that almost all philosophers focus on, meaning.
I think meaning is a problematic notion, and problematic in ways that will affect our attempt to understand how communication by language emerges.
But we haven't got all day, so I want to put this to one side for as long as possible.
 
 
\subsection{unit\_511}
 
\section{Does being able to think depend on being able to communicate with language?}
 
 
\subsection{slide-11}
Our question is, How do humans come to have abilities to communicate using words?
I want to step back from this question to consider an argument about thought and language.
Here's an argument to show that being able to think depends on (or is interdependent with) being able to communicate by language.
I take this argument from Davidson.
 
 
\subsection{slide-13}
This premise seems straightforward to me.
 
 
\subsection{slide-15}
Note that by think we mean desire, intend, wish, guess, believe ...
There's a quote on your handout in support of this.
 
 
\subsection{slide-19}
Why accept the second premise of the argument? Consider Davidson's reasoning:
 
 
\subsection{slide-20}
Why accept the second premise of the argument, that to be capable of having a false belief it is necessary to understand the possibility of false belief?
Here is Davidson's reasoning ...
 
 
\subsection{slide-23}
Here's an analogy for Davidson's argument.
I think it would be appropriate to say the yellers and stampers are making a mistake only if they themselves can see things that way.
I suggest that Davidson's claim is based on a similar intuition.
 
 
\subsection{slide-24}
Note that the analogy I've offered isn't an argument for Davidson's claim.
At most, it's a challenge to someone who rejects it.
If you reject the premise, then you have to explain what makes it appopriate to assign blame or apply correctness conditions.
 
 
\subsection{slide-25}
So here's the argument again.
So far we've considered the first two premises.
 
 
\subsection{slide-27}
What about the third premise?
This is really difficult.
I'd happily spend a lecture on it, but that would take us too far from the question that occupies us today.
So let me, for now at least, just put the claim in Davidson's own words ...
 
 
\subsection{slide-30}
Now just focus on the conclusion.
Recall that our overall question is, How could humans come to have abilities to communicate by language?
The conclusion of this argument provides one answer: Not by means of thinking.
To see what this rules out, consider this view from Higginbotham ...
 
 
\subsection{slide-33}
Higginbotham offers a beautifully simple answer to our question, How could humans come to have abilities to communicate by language?
The answer is that we figure out the meanings of words in just the way we figure other things out, like why Ayesha is so glum or who ate my breakfast.
But now recall Davidson's claim that If someone can think, she can communicate by language.
I take it that Davidson's claim is incompatible with the view that language acquisition involves forming hypotheses about the meanings of words at the outset.
Be careful about the incompatibility: it's not that forming hypotheses about the meanings of words can't be part of the process, at a later stage.
The problem is that this---forming hypotheses---can't happen before at least some linguistic competence is present.
So there is something essential missing from Higginbotham's picture, at least if Davidson is right.
 
 
\subsection{slide-34}
Let me be clear about what I'm saying.
 
 
\subsection{slide-35}
There are two claims ...
 
 
\subsection{slide-36}
I am saying that if the first is true, the second is false.
So if Davidson is right, we know something about how langauges are not acquired.
But I'm not saying that Davidson is right.
Indeed, there is a remarkable lack of convincing argument for (1) and claims like it.
 
 
\subsection{slide-37}
I am also saying that if the second is true, the first is false.
So one way to argue that Davidson's position must be wrong would be to argue that acquiring language involves thinking from the start.
 
 
\subsection{slide-39}
I want to consider this direction first.
Suppose Davidson is right. Can we give a plausible account of language acquisition?
 
 
\subsection{unit\_521}
 
\section{Training}
 
 
\subsection{slide-41}
For now I'm assuming that Davidson is right that somoene who can think communicate with language.
What account of language acquisition is consistent with this assumption?
A clue is given by Davidson ...
 
 
\subsection{slide-43}
So we might suppose that acquiring a language involves learning how to act without learning that anything is the case.
This is the general idea. How can we make it concrete?
 
 
\subsection{slide-44}
Our question is, How do humans come to have abilities to communicate using words?
Let's start with Bertrand Russell.
 
 
\subsection{slide-47}
But how does the environment determine habits and associations?
 
 
\subsection{slide-48}
Wittgenstein suggests that the habits are determined by training.
But how does this training work?
 
 
\subsection{slide-54}
But now what are these habits and associations?
 
 
\subsection{slide-55}
One answer is suggested by Quine.
So this is the picture.
For each word, there is a set of 'stimulations' in response to which an utterance of that word would be appropriate.
For instance, we might suppose there's a set of banana stimulations in response to which an utterance of the word 'banana' would be appropriate.
The child then comes to use the word 'banana' in response to the bananana-stimuluations by means of being trained.
She is rewarded for using 'banana' correctly or punished for using it incorrectly (or both) and so she gradually zeros in on the correct pattern of use.
 
 
\subsection{slide-57}
This seems to be approximately Davidson's own view.
 
 
\subsection{unit\_531}
 
\section{Understanding}
 
 
\subsection{slide-61}
For now I'm assuming that Davidson is right that anyone who can think communicate with language.
I've just been asking, What account of language acquisition is consistent with this assumption?
The answer was: some kind of training.
One problem for this view is that training does not seem sufficient.
To see why, let's ask paraphrase a famous question of Michael Dummett's (What do I know when I know a language?) and ask:
What do I know or understand when I can communicate with language?
What training gets you is just an ability to use words in certain circumstances.
But communicating with language seems to involve more than this.
It seems to involve understanding.
This, anyway, seems to be Davidson's view:
 
 
\subsection{slide-64}
This is just a hint, but I take it that Davidson is suggesting that communicating with language involves being able to understand what you've said.
And merely being trained in the ways Russel, Wittgenstein, Quine and Davidson envisage clearly doesn't provide any understanding at all.
I think this point is nicely made but Dummett.
 
 
\subsection{slide-65}
I want to outline Dummett's view in three quotes.
There's no argument here. I just think Dummett has hit on a datum about communication with language.
 
 
\subsection{slide-70}
I'll probably skip this, but it's a good example of the magic moment view.
It's also an interesting case where a serious philosopher says something which is empirically testable and where there no evidence for it but quite good evidence against it.
 
 
\subsection{slide-71}
So what am I saying?
Note that this is a point about suffiency. It's not an objection, but it is a lacuna.
And it's a lacuna that's hard to fill
It seems like, on this view, there has to be a magic moment when all that training somehow turns into the expression of thought.
Let me put this another way.
On this view, there is a gap between rat-like abilities to use words and thoughtful expression.
And no good account of how the gap could be crossed.
But this isn't the worst problem for the view. Creativing is a far worse problem ...
 
 
\subsection{unit\_541}
 
\section{Creativity}
 
 
\subsection{slide-74}
I'm assuming that Davidson is right that anyone who can think communicate with language.
 
 
\subsection{slide-82}
Try to imagine you have never communicated linguistically with anyone. You realise that other people interact much more easily that you can. You're sitting here and everyone else is concentrating or making notes and obviously getting something out of being here that you aren't. But what? What is it that they are doing and how are they doing it?
Some deaf children in North America are brought up in purely oral environments without any sign language and therefore don't experience language at all. These children invent their own sign languages, which are called homesigns. Their invented languages are not as rich as those of children who experience other people's languages, but they have many of the same features (Goldin-Meadow 2002, 2003). These deaf children have somehow worked out for themselves what linguistic communication is and they have found a way of doing it. They have invented languages with no prior experience of language, and they have invented languages in a modality that people around them barely use in linguistic communication. These linguistically isolated deaf children have answered in practice the questions that these lectures are about.
What are their languages like? Here are some examples ...
 
 
\subsection{slide-83}
“Pointing at the Present to Refer to the Non-Present. David points at the chair at the head of the dining room table in his home and then produces a “sleep” gesture to tell us that his father (who typically sits in that chair) is asleep in another room. He is pointing at one object to mean another and, in this way, manages to use a gesture that is grounded in the present to refer to someone who is not in the room at all” (Goldin-Meadow 2003: 74, figure 1)
 
 
\subsection{slide-84}
“Examples of Conventional Emblems Whose Meanings Are Not as Transparent as They Seem. In panel A, David is shown producing a “break” gesture. Although this gesture looks like it should be used only to describe snapping long thin objects into two pieces with the hands, all of the children used the gesture to refer to objects of a variety of sizes and shapes, many of which had not been broken by the hands. In panel B, Marvin is shown producing a “give” gesture. This gesture looks like it should mean “put something small in my hand,” but all of the children used it to request the transfer of an object, big or small, to a place that was not necessarily the child's hand. Thus, many of the gestures that the deaf children used were not as transparent in meaning as a quick glance would suggest” (Goldin-Meadow 2003: 76, figure 2).
 
 
\subsection{slide-85}
“David is holding a toy and uses it to point at a tray of snacks that his mother is carrying = snack (the tray is not shown in the drawing). Without dropping the toy, he jabs it several times at his mouth = eat. Finally, he points with the toy at me sprawled on the floor in front of him (not shown) = Susan” (Goldin-Meadow 2003: 110, figure 1).
 
 
\subsection{slide-86}
“With this long string of gestures, all produced before she relaxed her hands, Qing is indicating that swordfish can poke a person (proposition 1) so that the person becomes dead (proposition 2), that they have long, straight noses (proposition 3), and that they swim (proposition 4)” (Goldin-Meadow 2003: 170).
In more detail: “Complex Gesture Sentences. Qing [Chinese child] produces five distinct gestures that she combines into a single complex gesture sentence (that is, she produces the string of gestures without breaking her flow of movement). The five gestures are illustrated in this figure: Qing points at a picture of a swordfish (= swordfish). She jabs at her own chest as though piercing her heart (= poke-in-chest). She crooks her index finger and holds it in the air (this is an emblem in Taiwan that hearing speakers use to mean dead). She holds her index finger on her nose and extends it outward (= long-straight-nose). She wiggles her palm back and forth (= swim).” (Goldin-Meadow 2003: 171, figure 22)
 
 
\subsection{slide-87}
Can we say something about the general features of homesigns?
'gesture forms do not change capriciously with changing situations'
i.	‘The gestures are stable in form, although they needn’t be. It would be easy for the children to make up a new gesture to fit every new situation (and, indeed, that appears to be what hearing speakers do when they gesture along with their speech, cf. McNeill, 1992). But that’s not what the deaf children do. They develop a stable store of forms which they use in a range of situations-they develop a lexicon, an essential component of all languages (Goldin-Meadow, Butcher, Mylander, \& Dodge, 1994).’ \citep[p.\ 1389]{Goldin-Meadow:2002dq}
 
 
\subsection{slide-88}
'gesture--meaning pairs have arbitrary aspects within an iconic framework'
 
 
\subsection{slide-89}
'the gestures the children develop are composed of parts that form paradigms, or systems of contrasts. When the children invent a gesture form, they do so with two goals in mind-the form must not only capture the meaning they intend (a gesture-world relation), but it must also contrast in a systematic way with other forms in their repertoire (a gesture-gesture relation).' \citep[p.\ 1389]{Goldin-Meadow:2002dq}
 
 
\subsection{slide-95}
Children can create their own first languages.
I haven't explained the evidence for this claim here, but children in ordinary linguistic environments are also extremely creative from the beginning of their attempts to communicate.
What children do with words is, from the very beginning, purposively directed at sharing with others conscious attention to objects and events in their environment.
This means that there's no prospect at all of describing characteristically human antics without mentioning psychological notions like purpose, understanding, consciousness and attention.
The basic features of our mental lives can't be factored out of discussions by waffle about “forms of life” or “social practices” or “deontic scorekeeping”.
 
 
\subsection{slide-96}
How does this bear on our position? Recall ...
 
 
\subsection{slide-101}
So here's my challenge to Davidson and others who hold that anyone can communicate with language can think:
explain how someone could begin to create words without already being able to think.
As I've been explaining, the challenge arises because children who have no language and no significant experience of language can create languages of their own.
 
 
\subsection{slide-104}
For my part, I think it's probably time to drop the assumption.
Not because we've shown it's wrong, but because there's no good argument for it an a significant obstactle to accepting it.
So let's return to our overall question without that assumption.
(Recall that the question was, How could humans come to have abilities to communicate by language?)
 
 
\subsection{slide-105}
This is an aside. Take a break, don't listen. But things are worse for Davidson and others than you imagine.
 
 
\subsection{slide-106}
Recall Davidson's argument.
 
 
\subsection{slide-107}
So Davidson's view is that intentional action is impossible without language.
I think it's right to link intentional action to intention, and intention to thought.
So I think Davidson is right to this extent: if anyone who can think can communicate with language, then anyone who can act intentionally can communicate with language.
This underlines the difficulty of meeting that challange, of explaining how language creation gets going.
If Davidson is right, it must get going without either thinking or acting intentionally.
 
 
\subsection{unit\_551}
 
\section{Mapping words to concepts}
 
 
\subsection{slide-109}
Our question is, How do humans come to have abilities to communicate using words?
Let's make a fresh start and consider another approach.
 
 
\subsection{slide-110}
Does the view Wittgenstein is attacking sound like a mere caricature? Bloom explicitly endorses it, noting that ‘Augustine’s proposal is no longer seen as the goofy idea that it once was’ \citep[p.\ 61]{Bloom:2000qz}.
 
 
\subsection{slide-116}
This view on language acquisition is not new.
Eve Clark quotes a book from 1958 which apparently suggests that children learn words by formulating and testing hypotheses about their meanings:
 
 
\subsection{slide-117}
One consequence of this view ...
 
 
\subsection{slide-119}
The problem is ...
This is not enough for us to reject the Assumption. But it shows at least that the advocates of the Assumption have to explain how this case is possible and why they think there are no other such cases.
 
 
\subsection{slide-120}
Recall this idea:
 
 
\subsection{slide-121}
Incidentally, this isn't only Bloom's view, it's incredibly widespread.
Here's another example:
 
 
\subsection{slide-122}
I want to suggest that this idea is wrong because acquiring language is in significant part a creative process, as Eve Clark herself emphasises \citep[in][]{Clark:1993bv} .
Let me show you a particular case ...
 
 
\subsection{slide-123}
Is mapping illuminating?
Thanks to Roy Higginson's CHILDES data (1985) we can trace June's use of the novel word puttaputta over six months in fourteen conversations with her mother.
June's first recorded use of puttaputta occurs when she was fifteen months (1;3.0).
At first, her mother mistakenly takes “puttaputta” to mean Peter Piper and uses it as a noun:
 
 
\subsection{slide-125}
Note that the mother corrects June because at this point she mistakenly takes “puttaputta” to mean Peter Piper.
But that isn't what June has in mind, and she persists in using the term differently.
June persists in using “puttaputta” to mean something like tell me about this or read me this one and her mum quickly gets the hang of it (within the same conversation):
 
 
\subsection{slide-126}
In this first conversation, June uses “puttaputta” 20 times (3 of these are just “putta”). Most the other 15 words she uses appear only once and none appear more than twice.
So not only do infants coin new words, they will also persist in using them despite initial misunderstandings (even despite being 'corrected' by an adult) and they may rely heavily on their own words.
 
 
\subsection{slide-127}
In a later session (June is now seventeen months (1;5.0)) June continues to use “puttaputta” and to be understood as she intends:
 
 
\subsection{slide-129}
In this same session, June's mum uses “puttaputta” as a verb herself.
June continues to use “puttaputta” frequently in until around eighteen months (“puttaputta” occurs 45 times in a conversation recorded when June was 1;6.0) and then drops it abruptly (“puttaputta” doesn't appear in any of the seven conversations recorded over the next three months).
 
 
\subsection{slide-131}
Why am I telling you about June's use of puttaputta.
One thing to note, of course, is that June isn't learning to map a concept to a word.
If she is doing anything with a word-concept mapping, she is inventing and teaching it rather than learning it.
This is one ammendment to the claim that acquiring language depends on coming to know word-concept mappings.
But there is a second point I want to draw from 'puttaputta'.
June uses puttaputta purposively, to a particular end, to get others to read or interact with her in certain contexts.
But I'm not sure that this involves mapping the word to a concept.
Is mapping the word to the concept is the key to understanding what June is doing?
I'm not saying it's not; I'm just saying that there's a challenge to Bloom here.
 
 
\subsection{slide-132}
Is there an alternative picture of communication by language, one without mapping?
 
 
\subsection{slide-133}
So what am I saying?
 
 
\subsection{slide-135}
I've been considering the mapping idea, the idea that children have to establish a correspondence between words and concepts.
I agree that this the right way to think about some language learning; in particular, learning colour words like 'red' is well modelled in this way.
But I've offered two qualifications.
First, in at least some of these cases it may be that the concept comes after the word.
Second, while it's appropriate to think of abilities to use language as requiring word-concept mappings in some cases, like the case of 'red',
I used the example of 'puttaputta' to suggest that there are at least some cases of communiation with words which don't seem to involve learning or inventing word-concept mappings.
I'm not claiming to have shown that the mapping idea is wrong.
I merely want to offer you an open challenge: can we make sense of the mapping idea and do we need it?
 
 
\subsection{unit\_561}
 
\section{Summary}
 
 
\subsection{slide-140}
So there is a challenge for anyone who holds the Assumption.
The challenge is to explain how someone can start creating a language without being able to think
 
 
\subsection{slide-143}
So the first part was about trying to meet this constraint.
In the second part we made a fresh start and examined how you might answer the question if you dropped the constraint.
 
 
\subsection{slide-151}
So how do humans come to acquire abilities to communicate with language?
We've seen that there are challenges and objections to two prominent views.
Do I have to end with the conclusion that we don't know the first thing about it?
Or is there anything positive to say?
Maybe just a hint ...
 
 
\subsection{slide-152}
So are we merely saying this, or can we go further:
The quote is from Wittgenstein, … Philosophy of Psychology (1980: 2.24 [§128])
 
 
\subsection{unit\_571}
 
\section{Grice/Tomasello}
 
 
\subsection{slide-154}
In thinking about the alternative picture, I'm guided by two thoughts.
The first concerns social interaction.
So far we've considered two pictures of how children learn to communicate with words: training and reasoning to word-concept mappings.
Both pictures miss social interaction.
On one picture the infant is an outsider who might as well be observing those around her through a telescope.
On the other picture the child is a blank to be shaped by those around her.
There is no meaningful interaction between the infant and adults around her; or, if there is, it makes no difference to her development.
But could there be a role for social interaction in learning the meanings of words?
If you think about the examples of creativity --- homesigns and puttaputta --- this seems plausible.
Communication with words happens in the context of joint action, and it is a tool for joint action.
 
 
\subsection{slide-155}
The alternative picture is based on the ideas of Michael Tomasello and his colleagues.
 
 
\subsection{slide-157}
The second guiding thought in creating the alternative picture concens creativity.
It's standard to suppose that creativity is a side-issue and that learning from those around us is the norm.
This view is encouraged by the idea that language is fundamentally a system of conventions, and that coming to communicate with words is a matter of getting into those conventions.
Nothing could be further from the truth.
Acquiring the ability to communicate with words is an essentially creative process.
Conventions in the surrounding culture can be useful, but they are no more than useful accessories.
 
 
\subsection{slide-159}
So these are my two themes.
I'm going to create a view out of them.


%--- end paste
%--------------- 
 





\bibliography{$HOME/endnote/phd_biblio}



\end{document}