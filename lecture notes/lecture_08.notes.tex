 %!TEX TS-program = xelatex
%!TEX encoding = UTF-8 Unicode

%\def \papersize {a5paper}
\def \papersize {a4paper}
%\def \papersize {letterpaper}

%\documentclass[14pt,\papersize]{extarticle}
\documentclass[12pt,\papersize]{extarticle}
% extarticle is like article but can handle 8pt, 9pt, 10pt, 11pt, 12pt, 14pt, 17pt, and 20pt text

\def \ititle {Origins of Mind: Lecture Notes}
\def \isubtitle {Lecture 08}
%comment some of the following out depending on whether anonymous
\def \iauthor {Stephen A.\ Butterfill}
\def \iemail{s.butterfill@warwick.ac.uk% \& corrado.sinigaglia@unimi.it
}
%\def \iauthor {}
%\def \iemail{}
%\date{}

%\input{$HOME/Documents/submissions/preamble_steve_paper3}
\input{$HOME/Documents/submissions/preamble_steve_report2}

%no indent, space between paragraphs
\usepackage{parskip}

%comment these out if not anonymous:
%\author{}
%\date{}

%for e reader version: small margins
% (remove all for paper!)
%\geometry{headsep=2em} %keep running header away from text
%\geometry{footskip=1.5cm} %keep page numbers away from text
%\geometry{top=1cm} %increase to 3.5 if use header
%\geometry{bottom=2cm} %increase to 3.5 if use header
%\geometry{left=1cm} %increase to 3.5 if use header
%\geometry{right=1cm} %increase to 3.5 if use header

% disables chapter, section and subsection numbering
\setcounter{secnumdepth}{-1} 

%avoid overhang
\tolerance=5000

%\setromanfont[Mapping=tex-text]{Sabon LT Std} 


%for putting citations into main text (for reading):
% use bibentry command
% nb this doesn’t work with mynewapa style; use apalike for \bibliographystyle
% nb2: use \nobibliography to introduce the readings 
\usepackage{bibentry}

%screws up word count for some reason:
%\bibliographystyle{$HOME/Documents/submissions/mynewapa} 
\bibliographystyle{apalike} 


\begin{document}



\setlength\footnotesep{1em}






%--------------- 
%--- start paste


\title {Origins of Mind: Lecture Notes \\ Lecture 08}
 
\maketitle
 
 
\subsection{title-slide}
 
\section{A Puzzle about Pointing}
 
 
\subsection{slide-3}
Our overall project is to understand something about the emergence of knowledge of minds, objects, colours and the rest in human development.
My proposal is that we have to take two factors into account. One is core knowledge, the other is social interaction.
Our current project is to understand where abilities to communicate fit into this picture---to understand, that is, what role they might play in explaining how humans come to know things.
 
 
\subsection{slide-5}
*todo explain social intelligence
 
 
\subsection{slide-7}
Non-linguistic communication is special because it links to two of our ambitions.
First, we want to better understand how humans acquire abilities to communicate by language, and these seem to be built on their non-linguistic abilities.
Second, non-linguistic communication appears to be a paradigm manifestation of social intelligence, and perhaps one that doesn't already require any knowledge knowledge at all.
So non-linguistic communication may be an aspect of social intelligence that can explain the origins of knowledge in development.
 
 
\subsection{slide-8}
As I mentioned before, my overall plan is to work backwards from communication by language through non-linguistic communication and so to understanding action.
--------
 
 
\subsection{slide-9}
In this section my aim is to introduce a puzzle about pointing ...
 
 
\subsection{slide-10}
Why is this significant?
Because it implies two things:
First, it implies that infants' pointing is referential communication; that is, communication about an object.
(Contrast sharing a smile; we're communicating, but not necessarily referring.)
Second, it implies that infants have some understanding of joint engagement.
 
 
\subsection{slide-11}
Last time we looked at an experiment to show that this is true. You might remember it from this photo.
 
 
\subsection{slide-12}
By the way, this is related to the essay topic on shared intentionality.
Tomasello takes infants' pointing to be based on what he calls shared intentionality.
I don't want to pursue this here; I'm just mentioning it because you might want to draw on pointing in your seminar essays.
Instead I want to ask in a simpler way, how should we understand pointing as other non-linguistic forms of communication?
To approach this question, it's helpful to compare and contrast humans with other, nonhuman apes.
 
 
\subsection{slide-14}
Here is a beautifully simple story. Humans are highly cooperative from birth. Being highly cooperative enables them to communicate, linguistically and non-linguistically. Being able to communicate in turn enables them to acquire knowledge of minds, physical objects, numbers and the rest. This is a beautiful story and conceptually simple. Of course the challenge is to get from the story to the details.
But Tomasello doesn't stick with the simple story. Instead ...
 
 
\subsection{slide-17}
There is this additional element, shared intentionality. I don't understand what it is, but Tomasello and his colleagues are extraordinay scientsits so I think it's worth exploring.
That's it for now: all I want to do is highlight that our thinking about pointing is going to connect with questions about shared intentionality.
 
 
\subsection{slide-21}
But why suppose that ‘infant pointing is best understood … is best understood … as depending on … shared intentionality’?
It's goingt to take a while to answer this question ...
 
 
\subsection{slide-22}
So far it's been all very straightforward, but we're about to run into a puzzle.
The puzzle will help us to understand why someone might think that ‘infant pointing is best understood … is best understood … as depending on … shared intentionality’
 
 
\subsection{slide-28}
What does declaratively mean? Liszkowski and Tomasello call pointing declarative when its done to initiate joint engagement.
 
 
\subsection{slide-32}
So the discrepancy is not easily explained.
Comprehension is also missing ...
 
 
\subsection{slide-34}
This question is the puzzle. Or, rather, it's half of the puzzle. (The other half is about why infants don't point until they're around 11 months old.)
 
 
\subsection{slide-36}
In this experiment, we contrast failed reaches with pointing ...
Hare and Call (\citeyear{hare_chimpanzees_2004}) contrast pointing with a failed reach as two ways of indicating which of two closed containers a reward is in. Chimps can easily interpret a failed reach but are stumped by the point to a closed container.
You are the subjects. This is what you saw (two conditions). Your task was to choose the container with the reward.
Infants can do this sort of task, it's really easy for them \citep{Behne:2005qh}. (And, incidentally, they distinguish communicative points from similar but non-communicative bodily configurations.)
The pictures in the figure stand for what participants, who were chimpanzees, saw.
The question was whether participants would be able to work out which of two containers concealed a reward.
In the condition depicted in the left panel, participants saw a chimpanzee trying but failing to reach for the correct container.
Participants had no problem getting the reward in this case, suggesting that they understood the goal of the failed reach.
In the condition depicted in the right panel, a human pointed at the correct container.
Participants did not get the reward in this case as often as in the failed reach case, suggesting that they failed to understand the goal of the pointing action.
(Actually the apes were above chance in using the point, just better in the failed reach condition. Hare et al comment ;chimpanzees can learn to exploit a pointing cue with some experience, as established by previous research (Povinelli et al. 1997; Call et al. 1998, 2000), and so by the time they engaged in this condition they had learned to use arm extension as a discriminative cue to the food’s location' \citep[p.\ 578]{hare_chimpanzees_2004}.)
\footnote{ The contrast between the two conditions is not due merely to the fact that one involves a human and the other a chimpanzee. Participants were also successful when the failed reach was executed by a human rather than another chimpanzee \citep[][experiment 1]{hare_chimpanzees_2004}. }
\textbf{Note that} chimpanzees do follow the point to a container \citep[see][p.\ 6]{Moll:2007gu}.
 
 
\subsection{slide-41}
What we've said is about comprehending pointing.
But our question was, Why don't apes point?
I think we can answer both questions, the one about production and comprehension together.
I've taken the detour via comprehension only because I it's easier to see the answer in the case of comprehension.
 
 
\subsection{slide-42}
Here's what we already have about comprehending pointing gestures.
This explains why apes don't comprehend pointing gestures --- they don't know (2) or (3) or both.
But what can we say about why they don't point? Think about what would be involved in producing a pointing gesture.
 
 
\subsection{slide-44}
Here's parallel view about production.
So why don't apes point?
Because they don't know (2) or (3) or both.
 
 
\subsection{slide-45}
Tomasello also asks this question.
‘the specific behavioral form — distinctive hand shape with extended index finger — actually emerges reliably in infants as young as 3 months of age (Hannan \& Fogel, 1987). […] why do infants not learn to use the extended index finger for these social functions at 3 – 6 months of age, but only at 12 months of age?’ \citep[p.\ 716]{Tomasello:2007fi}
(Again Tomasello's answer involves shared intentionality: it's because they don't understand shared intentionality until around their first birthdays.)
We can answer this in the same way --- they don't understand communication.
 
 
\subsection{slide-46}
I want to say a tiny bit more on what is involved in understanding a pointing gesture.
Suppose that we are doing puzzle. Then if I point to a piece, I probably intend you to do something with it in the context of our activity.
By contrast, if we are tidying up, a point to the same object might mean something different.
So:
Comprehending pointing is not just a matter of locking onto the thing pointed to; it also involves some sensitivity to context \citep[see][]{Liebal:2010lr}.
This is nicely brought out in a study by Christina Liebel and others.
 
 
\subsection{slide-50}
18-month-olds can do this, but 14-month-olds can't. (Don't infer anything from null result.)
 
 
\subsection{slide-52}
‘Already by age 14 months, then, infants interpret communication cooperatively, from a shared rather than an egocentric perspective’ \citep[p.\ 269]{Liebal:2010lr}.
‘The fact that infants rely on shared experience even to interpret others’ nonverbal pointing gestures suggests that this ability is not specific to language but rather reflects a more general social-cognitive, pragmatic understanding of human cooperative communication’ \citep[p.\ 270]{Liebal:2010lr}.
 
 
\subsection{slide-53}
This makes sense of why chimpanzees don't point --- they don't understand communicative intention.
Note that this is an explanation which doesn't mention shared intentionality.
That's deliberate; sometimes Tomasello and colleagues answer this question by appeal to shared intentionality; I wanted to consider a simpler answer first and postpone thinking about shared intentionality for as long as possible.
Compare \citep[p.\ 516]{Tomasello:2010dy}: ‘they do not understand communicative intentions’
 
 
\subsection{slide-54}
But is this consistent with the findings that 12-month-old infants do point?
What is it to understand that by this pointing action another intends to communicate that?
 
 
\subsection{unit\_671}
 
\section{What is a communicative action?}
 
 
\subsection{slide-59}
What is a communicative action?
Why are we asking this question?
Let me try to explain it like this ...
 
 
\subsection{slide-60}
Recall what we said about comprehending and producing pointing gestures ...
 
 
\subsection{slide-62}
Both comprehending and producing require knowing things about communication.
To know something about communication you have to understand something about what it is, I suppose.
So to know what is involved in being able to produce and comprehend pointing gestures, we have to know something about what it is to communicate.
(This will also tell us what apes 6-month-old humans lack that prevents them from communicating with pointing gestures.)
 
 
\subsection{slide-63}
Reminder of the question.
 
 
\subsection{slide-64}
Let’s start with some simple examples of non-communicative actions.
Purely physical interaction.
 
 
\subsection{slide-67}
Why are these actions non-communicative? Ayesha intends her fake yawn to have an effect on Ben, but the effect is a physiological one. The response she wants from him is mechanical.
But there are also non-communicative actions which require a rational response from the people they’re directed at. For example:
 
 
\subsection{slide-70}
Note here that although Ayesha intends to provide Ben with misinformation, her action isn’t communicative.
Intuitively, there’s a difference between deliberately providing information or misinformation to someone and communicating with her (Grice 1989: 218).
So what makes an action communicative?
Paul Grice has a neat answer to this question.
He notes that we sometimes achieve things merely by letting other people know that we intend to achieve them.
Waving is one of the simplest examples:
 
 
\subsection{slide-73}
In this example, Ayesha’s goal is to get Ben to come over. Her means of achieving this is to get Ben to recognise that this is what she intends. So when she waves, her intention is that waving will let Ben know that she intends him to come over.
 
 
\subsection{slide-74}
You can achieve some things just by letting people know that you intend to achieve them. To achieve things in this way is to perform an act of communication.
 
 
\subsection{slide-76}
Note that, on this Gricean view, communicating involves having intentions about intentions.m
 
 
\subsection{slide-77}
First approximation: To communicate is to provide someone with evidence of an intention with the further intention of thereby fulfilling that intention
To communicate, then, is to attempt to fulfil an intention by making it manifest to someone else that you have this intention. If you’ve studied Grice you’ll know that his analysis of meaning led to a long and boring series of counterexamples and refinements, most of which shed no light on the nature of linguistic communication (Schiffer 1987). But what’s really important about Grice isn’t the attempt to analyse meaning: it’s his insight
 
 
\subsection{slide-78}
Recall the comprehension of pointing case; what is the confederate doing if she's pointing to inform?
 
 
\subsection{slide-88}
So to mean something by pointing, you have to have to have an intention about my recognition of an intention of yours concerning my reasons.
 
 
\subsection{slide-89}
One consequence of this would be that we can't appeal to non-linguistic communication in explaining the emergence of sophisticated forms of mindreading.
Why not? [Explain.]
Another consequence is an amazing discrepancy between knowledge of the mind and knowledge of the physical ...
 
 
\subsection{slide-90}
2.5-year-olds look longer when experimenter removes the ball from behind the wrong door, but don't reach to the correct door
That barriers stop solid objects is not reflected in children's practical reasoning until they are about two years earlier.
If we suppose that children who can point have an understanding of communication as Grice understands it, then we are saying that they have a fabulously sophisticated model of the mental around two years before they understand the first thing about physical causation.
Now this isn't an argument against the view that children have a Gricean understanding of communication.
But it does motivate looking for alternatives.
 
 
\subsection{slide-91}
But why are we even considering this idea, the idea that children have a Gricean understanding of communication?
Because it seems to be the view favoured by Tomasello, Carpenter, Liszkowski et al, who are leading experts on pointing ...
 
 
\subsection{slide-92}
Here's what I take to be the view of Tomasello and colleagues.
 
 
\subsection{slide-94}
No, I don't know what that is either. I've asked you to find out in the last essay for this course.
 
 
\subsection{slide-97}
If the theory of communicative action is correct, then the claims about development are incompatible.
 
 
\subsection{slide-98}
So, apparently, if Tomasello et al are right, I'm wrong about two things:
First, I'm wrong that knowlegde of others' minds first emerges around three or four years of age and that one-year-old infants have only core knowledge of mental states.
And, seccond, I'm probably also wrong to think that abilities to communicate (whether by language or not) could explain the emergence of knowledge of others' minds.
This is one reason for asking, What is a communicative action?
 
 
\subsection{slide-99}
What are the alternatives?
I want to mention two alternatives ...
 
 
\subsection{slide-101}
One alternative is inspired by opponents of the claim, inspired by Grice, that communication by language involves identifying utterer's intentions.
Inspired by Grice, you might think that this is fundamental to linguistic communication.
But philosophers like Dummett and (for different reasons) Millikan reject this view.
We might try to provide an account of pointing in which it's not fundamentally a matter of intention at all.
This would be a radical departure from the Gricean view about pointing. But there is another alternative, one which is less radical.
 
 
\subsection{slide-103}
Like Grice:
 
 
\subsection{slide-104}
We need to distinguish ulterior intentions from semantic intentions.
ulterior intentions: ‘intentions which lie as it were beyond the production of words … [such as] the intention of being elected mayor, of amusing a child, of warning a pilot of ice on the wings’ \citep[p.\ 298]{Davidson:1992pl}.
semantic intentions: intentions concerning the meaning of one’s utterance.
Why does this distinction matter?
Grice’s explicates meaning and communication in terms of ulterior intentions.
His project is to give a reductive analysis of these notions, meaning and communication.
Ulterior intentions are precisely what is needed for such an analysis of meaning.
This is because ulterior intentions ‘do not involve language, in the sense that their description does not have to mention language’ or any semantic concepts like meaning \citep[p.\ 298]{Davidson:1992pl}.
But, Davidson points out, we don’t have to attempt an analysis of meaning and communication.
After all, Grice’s analysis has been subject to plenty of counterexamples and objections \citep{Schiffer:1987zb}.
(Davidson objects that ‘it is not clear that these principles [Grice’s] are designed to handle the gamut of examples we find in literature’ \citep[p.\ 300]{Davidson:1992pl}.
\citet{Davidson:1991ic} discusses one literary example at length. He argues that ‘Joyce takes us back to the foundations and origins of communication; he puts us in the situation of the jungle linguist trying to get the hand of a new language and a novel culture, to assume the perspective of someone who is an alien or an exile’ \citep[p.\ 11]{Davidson:1991ic}.)
 
 
\subsection{slide-105}
This aim ‘assumes the notion of meaning’, but it is important because ‘it provides a purpose which any speaker must have in speaking, and thus constitutes a norm against which speakers and others can measure the success of their verbal behavior.’ \citep[p.\ 11]{Davidson:1994ol}
*todo: this is linked to how Davidson distinguishes first meaning from pragmatic bits; see 'meaning is a psychological concept v2' (for Martin Davies)
But how does this idea translate into a claim about what a pointing action is?
First consider the wave from earlier ...
 
 
\subsection{slide-106}
An example contrasting Grice and Davidson on the wave.
These intentions have a means-end ordering; the ulterior intention is further down the means-end chain.
Strictly speaking, that Ben should come over might not be the first meaning of the wave (so there are other options here).
 
 
\subsection{slide-108}
As mentioned before, Grice's view involves intentions about recognising intentions.
 
 
\subsection{slide-110}
By contrast, Davidson's view requires an intention about meaning.
What this involves depends, of course, on how we understand meaning.
But maybe there is a way of understanding meaning on which this is not too demanding. (I'm really not sure.)
 
 
\subsection{slide-111}
Now contrast Grice and Davidson on the pointing action from the Hare et al study, where you're supposed to take one of two containers.
Strictly speaking, that Ben should come over might not be the first meaning of the wave (so there are other options here).
 
 
\subsection{slide-114}
As before, there's a contrast in what must be intended and so what we're committing ourselves to in saying that infants can produce and comprehend informative pointing.
 
 
\subsection{slide-115}
The question was, Should we accept that pointing (and linguistic communication) involves intentions about intentions?
 
 
\subsection{slide-118}
Is this enough to save the view about development I wanted to offer. Maybe, maybe not …
 
 
\subsection{slide-120}
(*Davidson himself thinks all communication involves sophisticated insights into others’ minds …)
 
 
\subsection{slide-122}
We have to say what meaning or reference is such that infants understand it. I want to leave this as an open problem. When we talk about action next time I'll provide a model for dealing with this sort of problem.
 
 
\subsection{unit\_681}
 
\section{Words and Communicative Actions}
 
 
\subsection{slide-152}
Last time we focussed on how children get to associate words with their referents (or meanings, whatever exactly those turn out to be).
Here we discussed the idea that this comes about by training, and contrasted it with the idea that children identify the referents of words through the use of reason.
What I want to stress now is that getting the word-referent relation is only a part of what's needed to communicate by language.
This was brought out drammatically by the Hare and Tomasello task I mentioned earlier ...
 
 
\subsection{slide-153}
Recall this experiment in which Hare and Call (\citeyear{hare_chimpanzees_2004}) contrast pointing with a failed reach as two ways of indicating which of two closed containers a reward is in. Chimps can easily interpret a failed reach but are stumped by the point to a closed container.
\textbf{Note that} chimpanzees do follow the point to a container \citep[see][p.\ 6]{Moll:2007gu}.
Chimps do follow the point to the container, but they don't get the message.
 
 
\subsection{slide-154}
Tincoff and Jusczyk showed 6 month old infants two videos (not pictures: what you see here are stills from their videos) simultaneously.
While the videos were playing, the infants heard a word spoken. The word was either 'hand' or 'foot'.
Which video did they look at more?
 
 
\subsection{slide-155}
Here are Tincoff and Jusczyk's results.
They suggest that 6-month-olds can already associate some words with their referents.
But 6-month-old infants don't communicate, neither with words nor by pointing.
 
 
\subsection{slide-156}
Recall our simple description of what is involved in communication.
 
 
\subsection{slide-158}
What I'm suggesting is two very simple ideas. First, to be able to associate pointing gestures with their referents is not sufficient for understanding them; second, basic the pointing-referent associations may be grasped by non-communicators.
(Here the 'basic' qualifier is there because in some cases deciding on the referent may involve thinking about the agent and recepient, their knowledge and intentions.)
 
 
\subsection{slide-160}
And likewise for the case of linguistic communication.
 
 
\subsection{unit\_646}
 
\section{Syntax: Knowledge or Core Knowledge?}
 
 
\subsection{slide-162}
Given that humans, infant and adult, do represent facts concerning the syntax of languages, what grounds are there to deny that these representations are knowledge?"
In the previous lecture I answered this question too quickly. Afterwards someone pointed out that the considerations I had offered are not conclusive. So I wanted to return to this point briefly.
Why bother? I think it matters because (i) the case of syntax is well-understood; (ii) if representations of syntax are not knowledge, the case of syntax provides one possible model for theorising about core knowledge; and (iii) whereas in cases like objects, knowledge knowledge comes relatively early in development (maybe a year or two after core knowledge), in the case of syntax knowledge knowledge typically requires a course in linguistics.
This last point, (iii), makes it a paradigm case of development by rediscovery.
 
 
\subsection{slide-165}
Earlier I explained limited accessibility in terms of inferential integration with knowledge. So the two ideas are barely different.
 
 
\subsection{slide-166}
Aside: While we're thinking about Evans, it's worth mentioning parallels between his notion of tacit knowledge and the notion of core knowledge.
Evans' was one of the pioneers here, and his ideas have been pursued by other philosophers, so we can consider research on tacit knowledge to be a resource for understanding the notion of core knowledge.
 
 
\subsection{slide-167}
Evans characterisation of tacit knowledge involves two suggestions.
The first suggestion concerns similarities:
Tacit knowledge is analogous to belief at the level of input and output.
 
 
\subsection{slide-170}
Evans’ second suggestion concerns what distinguishes tacit knowledge from knowledge knowledge.
 
 
\subsection{slide-171}
Evans’ idea was ‘that while the applicability of the generality constraint is a necessary feature of propositional attitudes, states that are intuitively subdoxastic - particularly states of input systems - are not subject to that constraint.’ \citep[p.\ 146]{Davies:1986qv}
 
 
\subsection{slide-173}
It would take too long to explain the generality constraint here. If core knowledge matters to your project, I encourage you to read the chapter from Evans' book the Varieties of Reference in which the quote on your handout appears.
 
 
\subsection{slide-174}
Back from the aside, here's our conclusion.
 
 
\subsection{slide-175}
This is maybe a good time to consider a question running through these lectures.
 
 
\subsection{slide-177}
For instance, a module that detects causal relations contributes to development by providing us with knowledge that there are certain causal relations in our environment.
This knowledge can then be used for making inferences and guiding action, just as any other knowledge can:
 
 
\subsection{slide-182}
Why is this wrong? Because it ignores inaccessibility.
 
 
\subsection{slide-184}
I'll re-explain development by rediscovery with syntax as the illustration?
I don't think you should mention this in your essays, I'm still trying to work it out myself.
 
 
\subsection{unit\_701}
 
\section{Action}
 
 
\subsection{slide-193}
Our first question is, When do human infants first track goal-directed actions and not just movements?
In examining nonlinguistic communication, we've assumed that infants from around 11 months of age can produce and comprehend informative pointing.
This commits us to saying that they have
 
 
\subsection{slide-194}
\#source 'research/teleological stance -- csibra and gergely.doc'
\#source 'lectures/mindreading and joint action - philosophical tools (ceu budapest 2012-autumn fall)/lecture05 actions intentions goals'
\#source 'lectures/mindreading and joint action - philosophical tools (ceu budapest 2012-autumn fall)/lecture06 goal ascription teleological motor'
When do human infants first track goal-directed actions and not just movements?
Here's a classic experiment from way back in 1995.
The subjects were 12 month old infants.
They were habituated to this sequence of events.
 
 
\subsection{slide-196}
There was also a control group who were habituated to a display like this one but with the central barrier moved to the right, so that the action of the ball is 'non-rational'.
 
 
\subsection{slide-197}
For the test condition, infants were divided into two groups. One saw a new action, the other an old action.
Now if infants were considering the movements only and ignoring information about the goal, the 'new action' (movement in a straight line) should be more interesting because it is most different.
But if infants are taking goal-related information into acction, the 'old action' might be unexpected and so might generate greater dishabituation.
There was also a control group who were habituated to a display like this one but with the central barrier moved to the right, so that the action of the ball is 'non-rational'.
 
 
\subsection{slide-200}
The nonrational control group were habituated to a display like this one but with the central barrier moved to the right, so that the action of the ball is 'non-rational'.
 
 
\subsection{slide-201}
You might say, it's bizarre to have used balls in this study, that can't show us anything about infants' understanding of action.
But adult humans naturally interpret the movements of even very simple shapes in terms of goals.
So using even very simple stimuli doesn't undermine the interpretation of these results.
 
 
\subsection{slide-202}
Consider a related study by Woodward and colleagues.
(It's good that there is converging evidence from different labs, using quite different stimuli.)
 
 
\subsection{slide-204}
'Six-month-olds and 9-month-olds showed a stronger novelty response (i.e., looked longer) on new-goal trials than on new-path trials (Woodward 1998). That is, like toddlers, young infants selectively attended to and remembered the features of the event that were relevant to the actor’s goal.'
\citep[p.\ 153]{woodward:2001_making}
 
 
\subsection{slide-205}
Consider a further experiment by \citet{Csibra:2003jv}.
This is just like the first ball-jumping experiment except that here infants see the action but not the circumstances in which it occurs.
Do they expect there to be an object in the way behind that barrier?
 
 
\subsection{slide-207}
Why think about adults when you want to know about the origins of knowledge?
Because sometimes it's possible to identify adult abilities which are plausibly identical to infant abilities.
We saw an example of this in thinking about knowledge of minds, and in thinking about knowledge of colour.
In both cases, linking infants' competence to adults' competence allowed us to better understand the competence.
Can we also make such a link in the case of action?
 
 
\subsection{slide-208}
I already mentioned this classic study by Heider and Simmel.
This kind of study has been done with human adults quite a bit since then.
It is quite tempting to suppose that what we see here is automatic ascription of goals in adults.
 
 
\subsection{slide-211}
But is it automatic? Is it perceptual? Is there any evidence?
 
 
\subsection{slide-212}
\citet{Premack:1990jl} and \citet{Scholl:2000eq} all draw a parallel between causation and action.
(Premack's hypothesis that infants' understanding of intention is comparable to the understanding of causal interactions as shown by Michotte's launching stimuli (\citep{Premack:1990jl}; see also \citep{Premack:1997ek}).)
We saw (back in lecture 4) that there is quite convincing evidence that causal interactions are represented in perceptual proceses.
So if there is really a parallel, we could infer that relations between actions and goals are represented in perceptual processes too.
 
 
\subsection{slide-215}
Recall this habituation experiment.
they suggets that both
 
 
\subsection{slide-216}
The new Gergely et al experiment about action is based on the same habituation technique.
 
 
\subsection{slide-220}
If I had more time, I'd tell you a long story about speech and action.
Maybe some of us will get together another time and I can tell you that story then.
(*nb: \citep{zwickel:2010_interference} also argue for perceptual categories in perceiving actions, and do so by comparison with pop-out effects for orientations.)
 
 
\subsection{slide-221}
Subjects have to judge whether the dot is to the left or the right of the triangle (from their perspective).
If you think this is just a triangle, then it doesn't have a left or right so there's no congruent or incongruent.
But if you think the triangle performs goal directed actions, then in the figure on the top right, the dot is left of the triangle from your point of view but right of the triangle from its point of view.
Could there be altercentric interference?
 
 
\subsection{slide-224}
Here are the results.
When the triangle makes random movements: there's no difference in RT between congruent and incongruent conditions. (As you'd expect---this just shows that there's nothing wrong with the setup.)
(The experiment involved comparing neurotypical (ordinary) subjects with AS subjects; that's interesting but too complex for us so we'll just focus on the neurotypical subjects' perforamnce.)
Could there be altercentric interference?
 
 
\subsection{slide-227}
What can we conclude?
For adults, tracking information about goals makes one susceptible to interference from other other's perspective---it makes one susceptible to evaluting left and right from another's point of view.
This altercentric interference effect is a long way short of showing that the goal ascription is automatic or perceptual.
After all, such interference might be a consequence of thinking reflectively about the goals of the objects.
But it is at least a step in the direction in that it shows that, in adults, goal ascription can at least have effects on automatic processes.
(If automatic processes are informationally encapsulated, then this is evidence that goal ascription is automatic.)
(*Other evidence: \citep{Gao:2010,Teufel:2010})
 
 
\subsection{slide-228}
Would be nice to talk about this but we don't have time (could use some Brian Hare experiments; also Gomez?.)
 
 
\subsection{slide-229}
(The three months of age figure comes from \citet{Sommerville:2005te}.)m
 
 
\subsection{slide-232}
So it's maybe plausible there is core knowledge of action.
What we haven't done here (and can't do, as far as I know) is to show that infants lack knowledge knowledge of action.
So it's open to someone to deny that infants have core knowledge of action on the grounds that they have knowledge knowledge of it.
 
 
\subsection{slide-235}
Moving towards our next topic ...
But, What do infants understand of actions and their goals?
Or, as I like to put the quesiton, What model of action underpins infant cognition of action?
 
 
\subsection{slide-236}
What do infants understand of action? Intention?
\citet{Premack:1990jl} says yes.
But Geregely et al and Woodward et al say no.
 
 
\subsection{slide-237}
It isn't clear that this proposal can work (as introduced by Dennett, the intentional stance involves ascribing mental states), as these authors probably realised later, but the point about not representing mental states is good.
 
 
\subsection{slide-239}
What we've seen so far is (i) just what scientists think and (ii) almost entirely negative.
It's that infants don't represent intentions but do represent actions and their goals.
So we should ask, What do infants understand of actions and their goals?
Or, as I like to put the quesiton, What model of action underpins infant cognition of action?


%--- end paste
%--------------- 
 





\bibliography{$HOME/endnote/phd_biblio}



\end{document}