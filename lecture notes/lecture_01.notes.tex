 %!TEX TS-program = xelatex
%!TEX encoding = UTF-8 Unicode

%\def \papersize {a5paper}
\def \papersize {a4paper}
%\def \papersize {letterpaper}

%\documentclass[14pt,\papersize]{extarticle}
\documentclass[12pt,\papersize]{extarticle}
% extarticle is like article but can handle 8pt, 9pt, 10pt, 11pt, 12pt, 14pt, 17pt, and 20pt text

\def \ititle {Origins of Mind: Lecture Notes}
\def \isubtitle {Lecture 01}
%comment some of the following out depending on whether anonymous
\def \iauthor {Stephen A.\ Butterfill}
\def \iemail{s.butterfill@warwick.ac.uk% \& corrado.sinigaglia@unimi.it
}
%\def \iauthor {}
%\def \iemail{}
%\date{}

%\input{$HOME/Documents/submissions/preamble_steve_paper3}
\input{$HOME/Documents/submissions/preamble_steve_report2}

%no indent, space between paragraphs
\usepackage{parskip}

%comment these out if not anonymous:
%\author{}
%\date{}

%for e reader version: small margins
% (remove all for paper!)
%\geometry{headsep=2em} %keep running header away from text
%\geometry{footskip=1.5cm} %keep page numbers away from text
%\geometry{top=1cm} %increase to 3.5 if use header
%\geometry{bottom=2cm} %increase to 3.5 if use header
%\geometry{left=1cm} %increase to 3.5 if use header
%\geometry{right=1cm} %increase to 3.5 if use header

% disables chapter, section and subsection numbering
\setcounter{secnumdepth}{-1} 

%avoid overhang
\tolerance=5000

%\setromanfont[Mapping=tex-text]{Sabon LT Std} 


%for putting citations into main text (for reading):
% use bibentry command
% nb this doesn’t work with mynewapa style; use apalike for \bibliographystyle
% nb2: use \nobibliography to introduce the readings 
\usepackage{bibentry}

%screws up word count for some reason:
%\bibliographystyle{$HOME/Documents/submissions/mynewapa} 
\bibliographystyle{apalike} 


\begin{document}



\setlength\footnotesep{1em}






%--------------- 
%--- start paste



 
\title {Origins of Mind: Lecture Notes \\ Lecture 01}
 
\maketitle
 
 
\subsection{title-slide}
 
\section{The Question}
 
 
\subsection{slide-3}
There is a family of questions about the origins of mind that philosophers have been asking since Plato or before.
 
 
\subsection{slide-18}
Fodor mentions cognitive psychologists rather than philosophers. We will need to face up to the question of why philosophers are asking this question about the origins of knowledge, why is isn't just a scientific question. But that's something for later.
 
 
\subsection{slide-24}
This module is based on a simple question.
The question is,
How do humans come to know about---and to knowingly manipulate---objects, causes, words, numbers, colours, actions and minds?
 
 
\subsection{slide-25}
At the outset we know nothing, or not very much. (Like Lucas here.)
Sometime later we do know some things.
How does the transition occur?
 
 
\subsection{unit\_021}
 
\section{From Myths to Mechanisms}
 
 
\subsection{myths\_to\_mechanisms}
How do humans come to know about---and to knowingly manipulate---objects, causes, words, numbers, colours, actions and minds?
In a beautiful myth, Plato suggests that the answer is recollection.
Before we are born, in another world, we become acquainted with the truth.
Then, in falling to earth, we forget everything.
But as we grow we are sometimes able to recall part of what we once knew.
So it is by recollection that humans come to know about objects, causes, numbers and everything else.
 
 
\subsection{slide-28}
Leibniz explicitly endorses a version of Plato's view.
The view is subtler than it seems: we'll return to the subtelties later.
 
 
\subsection{slide-29}
Locke, as you probably know, was an empiricist. Here's his manifesto.
(Here I'm contrasting Plato's and Leibnz' nativism with Locke's empiricism.)
 
 
\subsection{slide-30}
This claim about colour isn't relevant yet but we'll return to it later.
 
 
\subsection{slide-31}
Spelke is blunt.
Spelke doesn't have exactly Locke vs Leibniz in mind here, but it's close enough.
The quote continues ‘humans are endowed neither with a single, general-purpose learning system nor with myriad special-purpose systems and predispositions. Instead, we believe that humans are endowed with a small number of separable systems of core knowledge. New, flexible skills and belief systems build on these core foundations.’
 
 
\subsection{slide-36}
The claim that we should shift from thinking about myths to mechanisms raises two questions.
First, what are the mechanism?
And, second, why suppose there is any role for philosophers rather than scientists?
 
 
\subsection{unit\_031}
 
\section{Inbetween mindless behaviour and thought}
 
 
\subsection{slide-38}
The question is how humans come to know about objects, words, thoughts and other things.
In pursuing this question we have to consider minds where the knowledge is neither clearly present nor obviously absent.
This is challenging because both commonsense and theoretical tools for describing minds are generally designed for characterising fully developed adults.
 
 
\subsection{object\_permanence}
The problem of describing ‘what is in between' (as Davidson puts it) is nicely illustrated by this question.
 
 
\subsection{slide-47}
These are the test events from Experiment 1 of Baillargeon et al's 1987 study.
 
 
\subsection{slide-54}
These are the results from Experiment 1 of Baillargeon et al's 1987 study.
 
 
\subsection{slide-57}
So the problem is that we seem to have conflicting answers to the question: one set of evidence says early, the other late.
Resolving the apparent conflict requires responding to an instance of Davidson's challenge and finding ways of describing phenomena in between mindless ignorance of unseen objects and adult-like knowledge of objects.
 
 
\subsection{slide-58}
So let's return to Davidson's point about what is in between mindless behaviour and thought.
 
 
\subsection{slide-61}
The paradox of early permanence provides us with a challenge of exactly this sort.
At least some of infants' abilities to represent objects they can't see don't involve knowledge or thought proper, but are neither entirely mindless. They do involve representation of some kind.
And this is why, despite scientific advances in understanding how humans come to know things, there is still work for philosophers.
 
 
\subsection{slide-62}
Hood and colleagues make a related point.
 
 
\subsection{slide-64}
This quote raises two issues
First, we should be cautious about the inference from separable systems to kinds of knowledge. (Think about modularity.)
 
 
\subsection{slide-67}
Second, come to think of it, shouldn't we be cautious here in talking about knowledge at all?
To see why, return one more time to the conflicting evidence about object permanence.
 
 
\subsection{slide-70}
At this point you might object. The question is about knowledge. But does either searching or distinguishing really provide evidence of knowing?
 
 
\subsection{slide-71}
You might insist that only abilities to talk about objects manifest evidence of knowledge.
It is a good question whether searching or distinguishing manifests knowledge.
We'll return to this point later.
 
 
\subsection{slide-73}
First note that we can duck this hard issue by changing from talk about knowledge to talk about representation.
Searching and distinguishing certainly seem to show that infants are representing objects they can't see.
Second note that whether or not the things manifested in searching and distinguishing are knowledge proper, they are surely things that matter for explaining how knowledge is eventually acquired.
 
 
\subsection{iceberg}
Charles \& Riverea note the conflict we have just been discussing.
Now you might think this conflict is just an isolated case.
Is the whole course going to be about infants' representations of objects? Not at all ...
 
 
\subsection{slide-76}
(Note that they mention violation-of-expectation experiments: we haven't yet come across violation-of-expectation experiments: I'll explain that notion later.)
 
 
\subsection{slide-79}
The problem that we see here generalises to other cases.
In what follows we will repeatedly encounter challenges of this form.
 
 
\subsection{slide-80}
To sum up so far, the question for this course is, How do humans come to know about---and to knowingly manipulate---objects, causes, words, numbers, colours, actions and minds?
I've been suggesting we can't answer it simply by appealing to nativism, empiricism or other grand myths.
Instead we need to focus on the particular mechanisms that are involved in different cases.
But then you might wonder, What philosophical questions arise here? Isn't this a narrowly pscyhological---and therefore scientific---issue?
The answer is no because thinking about how humans come to know things requires us to meet Davidson's challenge, to understand things that are neither mindless nor thought or knowledge but somewhere in between.
As Hood suggests in the quote I just showed you, this might involve rethinking what knowledge is.
 
 
\subsection{slide-81}
I hope I've given you a flavour of the approach we're going to take.
Good philosophy of mind has always been driven by scientific findings about the mind.
John Locke, David Hume as well as more recent philosophers like Jerry Fodor and Andy Clark all start with a deep understanding of the science of the mind.
But there is a difference.
Locke, Fodor and the rest are working on the big picture, trying to make explicit general features of the conceptual framework which scientists have more or less implicitly adopted.
By contrast, what I want us to do in this course is to look at specific problems that arise from the evidence,
and to provide philosophical tools for tackling this problem.
So you might say that whereas John Locke was trying to be the architect, we're trying more modestly to build the tools.
Now this might sound too modest to be interesting.
But, as we'll see, it turns out that attention to the details will give us new perspectives on some key philosophical issues about the nature of knowledge, perception and action.
 
 
\subsection{simple\_theory\_of\_the\_mind}
To illustrate how the modest approach of the tool maker might give us new perspectives on some key philosophical issues,
consider how the conflicting evidence about object permanence bears on the nature of minds quite generally.
 
 
\subsection{slide-83}
What the findings about infants' representations of objects they can't see show is this:
we must reject the simple theory of the mind.
 
 
\subsection{slide-84}
Recall the apparently contradictory pattern of findings.
The findings are apparently contradictory only if we accept the simple theory of the mind.
 
 
\subsection{slide-85}
So we must reject the simple theory.
Of course this doesn't follow just from the data I've mentioned so far.
But in these lectures we'll see the same sort of contradiction coming up in different domains, not just knowledge of objects but also number, colour, agency and others.
Now it's easy to say that the simple theory of the mind must be rejected.
But what are we to replace it with?
What the findings about infants' representations of objects they can't see show is this:
we must reject the simple theory of the mind.
 
 
\subsection{intentional\_stance}
We can make a similar point about Dennett's intentional stance.
The intentional stance is a theory about the nature of mental states like beliefs.
 
 
\subsection{slide-87}
There are two key components to the intentional stance.
The first component is a strategy.
 
 
\subsection{slide-88}
The second component is a claim about what it is to have a belief.
Now Dennett points out that his strategy works well for clams and thermostats.
But what about the infants?
If we consider just their pattern of looking times, it seems that the strategy does apply.
But if we consider their reaching behaviour, it seems that the strategy does not apply.
I think this reveals a deep problem for the strategy.
As with the simple theory of the mind, the Intentional Stance rests on the background assumption that individuals either have a particular beilef or else they do not have it.
But real minds are not so simple.
 
 
\subsection{slide-92}
You might object that this is unfair to Dennett and that the Intentional Stance provides a clear verdict on infants because they are not ‘reliably and voluminously predictable’.
But since he himself thinks the view applies to clams and theromstats, their simple behaviours must be sufficient for them to be ‘reliably and voluminously predictable’.
And in that case, so are infants looking times.
 
 
\subsection{slide-93}
We're going to try to understand how humans come to know about things by examining what developmental psychology tells us about the acquisition of knowledge.
This turns out to be a partly philosophical project because understanding the apparently conflicting evidence requires us to re-think notions like knowledge and representation.
In practice, this means looking carefully, and in detail, at the scientific evidence.
If you want to know how minds work, you have to start with the evidence.
 
 
\subsection{unit\_051}
 
\section{Two Breakthroughs}
 
 
\subsection{slide-102}
Infants can tackle physics, number, agents and minds thanks to a set of innate or early-developing abilities, often labelled “core knowledge”.
Several features distinguish core knowledge from adult-like understanding: its content is unknowable by introspection and judgement-independent; it is specific to quite narrow categories of event and does not grow by means of generalization; it is best understood as a collection of rules rather than a coherent theory; and it has limited application being usually manifest in the control of attention (as measured by dishabituation, gaze, and looking times) and rarely or never manifest in purposive actions such as reaching.
Preverbal infants manfiest a surprising range of social abilities. T
These include imitation, which can occur just days and even minutes after birth (Meltzoff \& Moore 1977; Field et al. 1982; Meltzoff \& Moore 1983), imitative learning (Carpenter et al. 1998), gaze following (Csibra \& Volein 2008), goal ascription (Gergely et al. 1995; Woodward \& Sommerville 2000), social referencing (Baldwin 2000) and pointing (Liszkowski et al. 2006).
Taken together, the evidence reveals that preverbal infants have surprisingly rich social abilities.
 
 
\subsection{unit\_061}
 
\section{Social Interaction}
 
 
\subsection{bold\_conjecture}
Here is a bold conjecture about how humans come to know things.
 
 
\subsection{vygostky\_hyp}
The challenge, of course, is to say *how* social interaction enables humans to come to know things.
 
 
\subsection{slide-115}
Does the view Wittgenstein is attacking sound like a mere caricature? Bloom explicitly endorses it, noting that ‘Augustine’s proposal is no longer seen as the goofy idea that it once was’ \citep[p.\ 61]{Bloom:2000qz}.
 
 
\subsection{slide-125}
Both pictures miss social interaction.
On one picture the infant is an outsider who might as well be observing those around her through a telescope.
On the other picture the child is a blank to be shaped by those around her.
There is no meaningful interaction between the infant and adults around her; or, if there is, it makes no difference to her development.
But could there be a role for social interaction in learning the meanings of words?
 
 
\subsection{slide-126}
Children acquiring language create their own words before they learn to use those of the adults around them.
Even where children have mastered a lexical convention, they will readily violate it in their own utterances in order to get a point across.
 
 
\subsection{slide-132}
Children with no experience of others' languages can create their own languages.
We know this from studies of profoundly deaf children brought up in purely oral environments and therefore without experience of language (Goldin-Meadow 2003; Kegl, Senghas and Coppola 1999; Senghas and Coppola 2001).
Individually or in groups these children invent their own signed languages.
These languages are not as rich as those of children with experience of other people's languages but they have all of the essential features of language including lexicons and syntax (Goldin-Meadow 2002, 2003).
The children invent gesture forms for words which they use with the same meanings in different contexts, they adopt standard orderings for combining words into sentences, and they use sentences in constructing narratives about past, present, future and hypothetical events. Thus one profoundly deaf child, Qing, describes how swordfish can poke a person so that she dies, and how they have long, straight noses and can swim (Goldin-Meadow 2003: 170).
 
 
\subsection{slide-133}
I started this section by contrasting two views on lexical acquisition, the shipwreck survivor view and the lab rat view.
We've seen that neither is the whole truth because children learn in part through a process of creation.
Why is this interesting?
The creative activity is a rational, goal-directed activity; so it's not merely training.
The creative activity does not obviously involve mapping words onto concepts, or even having the concepts.
(After all, the child might try out a pattern of use that others see as appropriate for a concept before the child actually has that concept.)
And this creativity with words involves social interaction: the aim is coordination between two or more individuals.
So: language acquisition is neither merely a matter of training, nor merely a matter of reasoning about the meanings of words.
Rather, it involves social interaction from the first utterances of words.
 
 
\subsection{unit\_071}
 
\section{Core knowledge}
 
 
\subsection{slide-135}
Earlier I explained that there is evidence that infants represent unseen objects from around four months.
At that point I mentioned just one experiment, Baillargeon's drawbridge study.
But there is a wide range of evidence all pointing in the same direction.
Here we'll look at just one more experiment.
 
 
\subsection{slide-137}
Here is another way of demonstrating object permanence.
The subjects were 4 month old infants.
They were shown a large object disappearing inside a small conatiner, or behind a narrow screen.
 
 
\subsection{slide-140}
The experiment was very simple.
All the experimenters did was measure how long infants looked in at the two events.
Infants looked longer at the narrow-occulder event.
 
 
\subsection{slide-141}
There was also a control condition.
In the control condition, infants saw a small rather than a large object.
 
 
\subsection{slide-142}
Here's the experimental condition again for comparison.
 
 
\subsection{slide-143}
And here's the control condition again.
 
 
\subsection{slide-144}
As you can see, there was a difference in looking times only in the experimental condition.
This experiment is interesting because it doesn't use habituation, as Baillargeon's earlier drawbridge experiment did.
 
 
\subsection{slide-145}
How should we interpret these results?
 
 
\subsection{slide-151}
Why is it core knowledge?
Part of the answer is that it controlls intelligent behaviours like looking.
Use of the term ‘knowledge’ is also supposed to contrast with perception.
But we can go deeper by considering a theory about object perception.
 
 
\subsection{slide-152}
How do infants or adults discern where one object begins and another ends?
The way they are ordinarily arranged in space, so that one occludes parts of another, prevents us from doing this in any simple way.
 
 
\subsection{slide-153}
Spelke proposes that our ability to identify objects depends on core knowledge of some principles (Spelke 1990).
Why is it core knowledge?
 
 
\subsection{slide-158}
How do these principles enable us to discern when one object ends and another begins?
 
 
\subsection{slide-160}
So I was asking why infants' representations of unseen objects should move us to say that they have core knowledge of objects.
Part of the answer was just that these representations guide intelligent behaviour and are not straightforwardly perceptual.
But Spelke's theory of object perception offers a deeper account.
The representations of objects are based on inference, or an inference-like process, which relies on approximately true principles.
 
 
\subsection{unit\_081}
 
\section{Linking Core Knowledge and Social Interaction}
 
 
\subsection{slide-162}
Here's a hypothesis linking core knowledge and social interaction:
(a) without core knowledge, we can't make sense of early forms of social interaction;
and (b) without social interaction, we can't explain the transition from core knowledge to knowledge proper.
 
 
\subsection{slide-163}
To flesh this out, I want to introduce the idea that cognitive development is rediscovery.
(But not exactly as Plato's myth had it.)
There are a series of ways in which knowledge or its prototypes are manifested.
We can ask how humans get from one manifestation to the next.
I want to suggest that it involves a process of rediscovery.
I don't think it's a case of partial knowledge becoming gradually more complete.
Rather, I think the representations at earlier stages shape subject's body, behaviour and attention in ways that facilitate discovery at the next stage.
And, most importantly, they enable increasingly right forms of social interaction.
It is these social interactions---together with the bodily, behavioural and mental changes---that enable subsequent re-discoveries.
So, on this view, the role of some early-developing abilities in explaining the later acquisition of conceptual understanding does not involve direct representational connections; rather the early developing abilities facilitate or enable social interaction and influence attention and inform behaviour, and these influences facilitate development.
 
 
\subsection{slide-164}
To make this vague idea slightly more concrete, let me zoom in.
I think having verbal labels for things sometimes helps with acquiring concepts of them.
Now this sounds paradoxical. Doesn't having a label for something mean being able to label correctly? And how could you label correctly without the corresponding concept?
My suggestion is that having core knowledge is not having a concept (many disagree);
but core knowledge could underpin your correct use of a label.
Now labels are acquired through social interaction, or so I suggested earlier.
Hence the picture.
Now this picture needs two qualifications. First, it's missing some details, and these details will vary from case to case (what's true of knowledge of objects might not be true of knowledge of number). Second, the picture might be completely wrong.
But even if the picture is wrong, I'll bet that social interaction and core knowledge are both essential for explaining how humans come to know things.
So I don't claim to know how these two are necessary, only that they are.
In fact my aim isn't primarily to explain to you how these two factors explain the origins of mind.
Instead my hope is this.
I'll give you the background on social interaction and core knowledge.
And you'll tell me how these two (or perhaps other factors) are involved in explaining how humans come to know things about objects, colours, actions, numbers and the rest.
 

%--- end paste
%--------------- 
 





\bibliography{$HOME/endnote/phd_biblio}



\end{document}