 %!TEX TS-program = xelatex
%!TEX encoding = UTF-8 Unicode

%\def \papersize {a5paper}
\def \papersize {a4paper}
%\def \papersize {letterpaper}

%\documentclass[14pt,\papersize]{extarticle}
\documentclass[12pt,\papersize]{extarticle}
% extarticle is like article but can handle 8pt, 9pt, 10pt, 11pt, 12pt, 14pt, 17pt, and 20pt text

\def \ititle {Origins of Mind: Lecture Notes}
\def \isubtitle {Lecture 02}
%comment some of the following out depending on whether anonymous
\def \iauthor {Stephen A.\ Butterfill}
\def \iemail{s.butterfill@warwick.ac.uk% \& corrado.sinigaglia@unimi.it
}
%\def \iauthor {}
%\def \iemail{}
%\date{}

%\input{$HOME/Documents/submissions/preamble_steve_paper3}
\input{$HOME/Documents/submissions/preamble_steve_report2}

%no indent, space between paragraphs
\usepackage{parskip}

%comment these out if not anonymous:
%\author{}
%\date{}

%for e reader version: small margins
% (remove all for paper!)
%\geometry{headsep=2em} %keep running header away from text
%\geometry{footskip=1.5cm} %keep page numbers away from text
%\geometry{top=1cm} %increase to 3.5 if use header
%\geometry{bottom=2cm} %increase to 3.5 if use header
%\geometry{left=1cm} %increase to 3.5 if use header
%\geometry{right=1cm} %increase to 3.5 if use header

% disables chapter, section and subsection numbering
\setcounter{secnumdepth}{-1} 

%avoid overhang
\tolerance=5000

%\setromanfont[Mapping=tex-text]{Sabon LT Std} 


%for putting citations into main text (for reading):
% use bibentry command
% nb this doesn’t work with mynewapa style; use apalike for \bibliographystyle
% nb2: use \nobibliography to introduce the readings 
\usepackage{bibentry}

%screws up word count for some reason:
%\bibliographystyle{$HOME/Documents/submissions/mynewapa} 
\bibliographystyle{apalike} 


\begin{document}



\setlength\footnotesep{1em}






%--------------- 
%--- start paste


\title {Origins of Mind: Lecture Notes \\ Lecture 05}
 
\maketitle
 
 
\subsection{title-slide}
 
\section{Knowledge of Mind}
 
 
\subsection{slide-3}
The challenge is to explain the emergence, in evolution or development, of mindreading.
Let me explain.
 
 
\subsection{slide-6}
\textit{Mindreading} is the process of identifying mental states and purposive actions as the mental states and purposive actions of a particular subject.
 
 
\subsection{slide-7}
Researchers sometimes use the term ‘theory of mind’.
‘In saying that an individual has a theory of mind, we mean that the individual imputes mental states to himself and to others’
\citep[p.\ 515]{premack_does_1978}
 
 
\subsection{slide-9}
So the challenge is to explain the emergence of mindreading.
You know (let's say) that Ayesha belives Beatrice is in the library.
Humans are not born knowing individuating facts about others' beliefs.
How do they come to be in a position to know such facts?
Meeting this challenge initially seems simple.
But, as you'll see, we quickly end up with a puzzle.
I think this puzzle requires us to rethink what is involved in having a conception of the mental.
 
 
\subsection{slide-12}
(NB: The figure is not Wimmer \& Perner's but drawn from their data.)
 
 
\subsection{slide-13}
There's been some stuff in the press recently about bad science, mainly some dodgy methods and failures to replicate.
 
 
\subsection{slide-14}
So you'll be pleased to know that a meta-study of 178 papers confirmed Wimmer \& Perner's findings.
 
 
\subsection{slide-15}
This meta-study also brings together some nice findings.
One is that you get essentially the same results whether you ask children about others' beliefs or their own beliefs.
Children literally do not know their own minds.
 
 
\subsection{slide-16}
What happens if we involve the child by having her interact with the protagonist?
The task becomes easier for children of all ages, but the transition is essentially the same (participation does not interact with age \citealp[pp.\ 665-7]{Wellman:2001lz}).
 
 
\subsection{slide-17}
Finally, although there are some cultural differences, you get the same transition in seven diferent countries.
 
 
\subsection{slide-18}
The challenge is to explain the emergence, in evolution or development, of mindreading.
Let me explain.
So our challenge was to explain the emergence of mindreading.
At this point, up until around, it seemed quite straightforward to most researchers.
We seemed to know that children are unaware of mental states until around four years.
And a lot of studies looked at which factors affect their acquiring this awareness.
These studies showed that executive function, language and rich forms of social interaction are all important.
All of this supported something like the story that Sellars tells in his famous Myth of Jones.
 
 
\subsection{slide-19}
*todo*: describe Sellars' myth
 
 
\subsection{slide-20}
But there was a big surprise in store for us.
 
 
\subsection{unit\_411}
 
\section{Mindreading: First Puzzle}
 
 
\subsection{slide-22}
There is a puzzle about when humans can first know individuating facts about others' beliefs.
To understand the origins of this knowledge we need to understand the puzzle.
So I'm going to reveal the puzzle to you. But let me start with a bit of background.
 
 
\subsection{slide-23}
What got me hooked philosophical psychology, and on philosophical issues in the development of mindreading in particular was a brilliant finding by Wendy Clements who was Josef Perner's phd student.
These findings were carefully confirmed \citep{Clements:2000nc,Garnham:2001ql,Ruffman:2001ng}.
Around 2000 there were a variety of findings pointing in the direction of a confict between different measures.
These included studies on word learning \citep{Carpenter:2002gc,Happe:2002sr} and false denials \citep{Polak:1999xr}.
But relatively few people were interested until ...
 
 
\subsection{slide-75}
The challenge is to explain the emergence, in evolution or development, of mindreading.
Initially it looked like this was going to be relatively straightforward and involve just language, social interaction and executive function.
So a Myth of Jones style story seemed viable.
But the findings of competence in infants of around one year of age changes this.
These findings tell us that not all abilities to represent others' mental states can depend on things like language.
Further, we have a puzzle.
The puzzle is how to reconcile infants' competence with three-year-olds' failure.
 
 
\subsection{slide-76}
The puzzle is a little bit like the puzzle we had in the case of knowledge of physical objects.
But it's also different.
In the case of physical objects, the conflict was between measures involving looking and measures involving searching.
In this case it's different, because on the infant side there is not just looking but also acting (e.g. helping) and even communicating.
 
 
\subsection{slide-77}
Can we solve the puzzle by appeal to core knowledge (or 'modularity')?
The difference in measures is a hopeful sign that we can.
But the fact that representations of others' minds influence 1-year-olds' actions (e.g. in communicating and helping) complicates things because we imagine modules as inferentially isolated from practical reasoning.
Looking at a further puzzle will help us.
 
 
\subsection{slide-78}
Here's the first puzzle one more time.
 
 
\subsection{unit\_421}
 
\section{Mindreading: Second Puzzle}
 
 
\subsection{unit\_431}
 
\section{Modules and Efficiency}
 
 
\subsection{slide-98}
We can resolve both puzzles by appeal to the idea that there are modular \& non-modular mindreading processes. (Cf. physical objects).
 
 
\subsection{slide-99}
Explain how this works in the case of each puzzle.
*todo* modify the slide to illustrate the solution
 
 
\subsection{slide-100}
To invoke modularity, we need to understand how mindreading could be (i) automatic and (ii) present in pre-linguistic infants with limited working memory \& executive function; both (i) and (ii) mean we need to understand how it could be cognitively efficient. (The situation is a bit like this: we want to say you can perceive others mental states; but on the face of it, mental states are exactly the sort of things that are not available to perception.)
 
 
\subsection{slide-101}
We saw earlier that mindreading in four year olds and adults is cognitively demanding.
And there's good reason to think that it should be.
If anything should consume scarce cognitive resources ...
Now appeal to modularity doesn't explain how mindreading might somehow be efficient.
Suppose someone could find prime factors incredibly quickly.
It wouldn't be a satisfying explanation to just say that she had a module for finding prime factors.
We'd also need an algorithm that her module could be implementing consistently with her performance.
So (a) efficiency points to modularity but (b) efficiency requires explanation and (c) gesturing at modularity doesn't explain efficiency.
To see how mindreading could be cognitively efficient, we need to reject a dogma.
 
 
\subsection{unit\_441}
 
\section{Minimal Theory of Mind}
 
 
\subsection{unit\_451}
 
\section{Signature Limits Generate Predictions}
 
 
\subsection{slide-144}
*todo*
- theme B : the puzzles about development (A- \& B-tasks) and automaticity. We've earned the right to solve them by appeal to modularity. (NB: it's modularity rather than mTm that explains the discrepancy; mTm is important because (i) it explains efficiency; and (ii) it generates predictions via signature limits)
- theme B-1 : relation between infant and adult competence --- infant competence retained in adults (matching signature limits from Low \& Watts), as in the cases of colour (and probably physical objects too);
- theme A: explain the origins of knowledge of others minds : development as rediscovery. There is a modular capacity ( = core knowledge). But this doesn't lead to adult-like understanding for years, and the acquisition of adult-like understanding hinges on language; may involve completely different model of mental states.
*todo* especially illustrate the two views about relation between infant and adult capacity (use All Souls' slides!)




%--- end paste
%--------------- 
 





\bibliography{$HOME/endnote/phd_biblio}



\end{document}