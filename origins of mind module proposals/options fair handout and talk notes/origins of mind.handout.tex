%!TEX TS-program = xelatex
%!TEX encoding = UTF-8 Unicode

\documentclass[12pt]{extarticle}
% extarticle is like article but can handle 8pt, 9pt, 10pt, 11pt, 12pt, 14pt, 17pt, and 20pt text

\def \ititle {Origins of Mind: Philosophical Issues in Cognitive Development}
\def \isubtitle {}
\def \iauthor {Stephen A. Butterfill}
\def \iemail{s.butterfill@warwick.ac.uk}
\date{}

%for strikethrough
\usepackage[normalem]{ulem}

\input{$HOME/Documents/submissions/preamble_steve_handout}

\bibpunct{}{}{,}{s}{}{,}  %use superscript TICS style bib
%remove hanging indent for TICS style bib
%TODO doesnt work
\setlength{\bibhang}{0em}
%\setlength{\bibsep}{0.5em}


%itemize bullet should be dash
\renewcommand{\labelitemi}{$-$}

% disables chapter, section and subsection numbering
\setcounter{secnumdepth}{-1} 


\begin{document}

\begin{multicols}{3}

\setlength\footnotesep{1em}


\bibliographystyle{newapa} %apalike

%\maketitle
%\tableofcontents




\begin{center}
{\Large
\textbf{Origins of Mind:}
\\ {Philosophical Issues in Cognitive Development}
}


<s.butterfill@warwick.ac.uk>

\end{center}



\section{The Challenge}
Explain how humans come to know about objects, causes, words, numbers, colours, actions and minds.

\section{What will I learn about?}
Recent scientific breakthroughs about the emergence of minds in development, and the philosophical issues arising from these.

Also issues like innateness, modularity and pre-linguistic cognition.

\section{How is it organised?}
The course will be organised around domains of knowledge.
The topics to be covered are:
\begin{enumerate}
\item Social Interaction without Words
\item Objects and How They Interact
\item Numbers
\item Seeing and Talking about Colours
\item Words and Other Communicative Tools
\item Actions: Teleology and Motor Awareness
\item Mindreading
\end{enumerate}

The topics are chosen so that each set of developmental findings is linked to a philosophical issue. 
For instance, research on knowledge of objects gives bite to questions about modularity and tacit knowledge.  


\section{How will it be taught?}
I’m planning sixteen lectures (two lectures most weeks) and five discussion meetings plus 4--5 small group tutorials involving a mix of presentations, essays and peer review.  


\section{What will I have to do?}
Write 3--5 short essays over the term (on which you’ll get feedback).  

Read lots of papers, including quite a bit of developmental psychology.

Learn to integrate philosophy with psychology.

\section{How will it be assessed?}
As an undergraduate, you can either write a 2500 word essay or take a 2 hour exam.
MA students write a 5000 word essay.


%\section{What are the requirements?}
%I assume you’ve done some philosophy or some psychology.



\section{Is there a web page?}
http://origins-of-mind.butterfill.com

\section{Can you suggest some reading?}
There’s a list on the back of this handout.

%\vfill 
%
%\columnbreak



\section{Reading Suggestions by Topic}

\subsection{1. Social Interaction without Words}
Could social interaction enable cognitive development? 
\citep{Moll:2007gu,
meltzoff:1977_imitation,
Csibra:2008be,
Baldwin:2000qq,
Liszkowski:2008al,
tomasello:2008origins}


\subsection{2. Objects and How They Interact}
\label{ch:objects}
When can infants first know things about objects they aren't perceiving?\citep{Meltzoff:1998wp,
baillargeon:1985_object,
Fodor:1983dg}



\subsection{3. Numbers}

How might abilities based on core knowledge enable the emergence in development of knowledge proper?\citep{
xu:2000_large,
wynn:1992_addition,
Fodor:1981ep,
carey:2009_origin}



\subsection{4. Seeing and Talking about Colours}
How do children acquire colour concepts and colour words---concepts and words for red, blue and green, say?\citep{Soja:1994np,
franklin:2010_hemispheric,
Franklin:2005hp}



\subsection{5. Words and Other Communicative Tools}
What comes first in development, knowledge or language?\citep{Wittgenstein:1974dk,
Bloom:2000qz,
Senghas:2001zm,
Goldin-Meadow:2003pj} 




\subsection{6. Actions: Teleology and Motor Awareness}
Which events do infants take to be actions?
And how do they understand the relation between actions and the goals to which they are directed?\citep{Bratman:1987xw,Csibra:1998cx,
rizzolatti_functional_2010,
bekkering:2000_imitation}

\subsection{7. Mindreading}
What is involved in representing belief?\citep[]{Onishi:2005hm, Southgate:2007js,kovacs_social_2010,
Davidson:1990du}

 \setlength{\bibsep}{3pt}
\footnotesize 
\bibliography{$HOME/endnote/phd_biblio}

\end{multicols}

\end{document}