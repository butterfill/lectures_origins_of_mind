%!TEX TS-program = xelatex
%!TEX encoding = UTF-8 Unicode

\documentclass[12pt]{extarticle}
% extarticle is like article but can handle 8pt, 9pt, 10pt, 11pt, 12pt, 14pt, 17pt, and 20pt text

\def \ititle {Origins of Mind}
 
\def \isubtitle {Lecture 02}
 
\def \iauthor {Stephen A. Butterfill}
\def \iemail{s.butterfill@warwick.ac.uk}
\date{}

%for strikethrough
\usepackage[normalem]{ulem}

\input{$HOME/Documents/submissions/preamble_steve_handout}

%\bibpunct{}{}{,}{s}{}{,}  %use superscript TICS style bib
%remove hanging indent for TICS style bib
%TODO doesnt work
\setlength{\bibhang}{0em}
%\setlength{\bibsep}{0.5em}


%itemize bullet should be dash
\renewcommand{\labelitemi}{$-$}

\begin{document}

\begin{multicols}{3}

\setlength\footnotesep{1em}


\bibliographystyle{newapa} %apalike

%\maketitle
%\tableofcontents




%--------------- 
%--- start paste


\def \ititle {Origins of Mind}
 
\def \isubtitle {Lecture 03}
 
 
 
\
 
 
 
\begin{center}
 
{\Large
 
\textbf{\ititle}: \isubtitle
 
}
 
 
 
\iemail %
 
\end{center}
 
 
 
\section{Objects vs Features}
 
Knowledge of objects depends on abilities to (i) segment objects, (ii) represent them as persisting and (iii) track their interactions.
 
The question for this lecture is, How do humans come to meet the three requirements on knowledge of objects?
 
 
 
\section{Segmentation and the Principles of Object Perception}
 
\textbf{Principles of Object Perception \citep{Spelke:1990jn}}
 
cohesion—‘two surface points lie on the same object only if the points are linked by a path of connected surface points’
 
boundedness—‘two surface points lie on distinct objects only if no path of connected surface points links them’
 
rigidity—‘objects are interpreted as moving rigidly if such an interpretation exists’
 
no action at a distance—‘separated objects are interpreted as moving independently of one another if such an interpretation exists’

\vfill
\columnbreak
 
What is the status of these principles?
\begin{enumerate}
\item We (as perceivers) start with a cross-modal representation of three-dimensional perceptual features which includes their locations and trajectories.

\item Our task is to get from these representations of features to representations of objects.

\item \emph{Descriptive component} We do this as if in accordance with certain principles (cohesion, boundedness, rigidity, and no action at a distance).

\item \emph{Explanatory component} We acquire representations of objects because we apply the principles to representations of features and draw appropriate inferences.

\end{enumerate}

 
‘Chomsky’s nativism is primarily a thesis about knowledge and belief; it aligns problems in the theory of language with those in the theory of knowledge. Indeed, as often as not, the vocabulary in which Chomsky frames linguistic issues is explicitly epistemological. Thus, the grammar of a language specifies what its speaker/hearers have to know qua speakers and hearers; and the goal of the child’s language acquisition process is to construct a theory of the language that correctly expresses this grammatical knowledge.’
\citep[p.\ 11]{Fodor:2000cj}
 
\subsection{The simple view}
 
The principles of object perception are things that we know, and we generate expectations from these principles by a process of inference.
 
 
 
\section{From Segmentation to Permanence}
 
\emph{Principle of continuity} An object traces exactly one connected path over space and time \citep[p.\ 113]{spelke:1995_spatiotemporal}.
 
\begin{center}
 
\includegraphics[scale=0.3]{../www.slides/src/files/img/spelke_1995_fig1.neg.png}
 
\end{center}
 
 
 \vfill
\columnbreak

\section{Permanence}
 
Object permanence is found in nonhuman animals including
 
\begin{enumerate}
 
\item monkeys \citep{santos:2006_cotton-top}
 
\item lemurs \citep{deppe:2009_object}
 
\item crows \citep{hoffmann:2011_ontogeny}
 
\item dogs and wolves \citep{fiset:2013_object}
 
\item cats \citep{triana:1981_object}
 
\item chicks \citep{chiandetti:2011_chicks_op}
 
\item dolphins \citep{jaakkola:2010_what}
 
\item ...
 
\end{enumerate}
 
Interpreting violation-of-expectation experiments:
 
‘evidence that infants look reliably longer at the unexpected than at the expected event is taken to indicate that they (1) possess the expectation under investigation; (2) detect the violation in the unexpected event; and (3) are surprised by this violation. The term surprise is used here simply as a short-hand descriptor, to denote a state of heightened attention or interest caused by an expectation violation.’
\citep[p.\ 168]{wang:2004_young}
 
‘To make sense of such results [i.e. the results from violation-of-expectation tasks], we … must assume that infants, like older learners, formulate … hypotheses about physical events and revise and elaborate these hypotheses in light of additional input.’
\citep[p.\ 329]{Aguiar:2002ob}
 
‘action demands are not the only cause of failures on occlusion tasks’
\citep[p.\ 291]{shinskey:2012_disappearing}
 
‘These dissociations cast doubt on the view that visual search and preferential looking depend on a single mechanism […] operating in accord with a single set of principles. [...]
‘infants’ ability to perceive object identity over occlusion, as assessed in preferential looking tasks, and to track objects visually, as assessed in visual search tasks, do not draw on a single system of knowledge. [...]
 
‘In Fodor’s (1983) terms, visual tracking and preferential looking each may depend on modular mechanisms.’
\citep[p.\ 137]{spelke:1995_spatiotemporal}
 
 
 
\section{Causal Interactions}
 
‘object perception reflects basic constraints on the motions of physical bodies …’
\citep[p.\ 51]{Spelke:1990jn}
 
‘A single system of knowledge … appears to underlie object perception and physical reasoning’
\citep[p.\ 175]{Carey:1994bh}
 
‘A similar permanent dissociation in understanding object support relations might exist in chimpanzees. They identify impossible support relations in looking tasks, but fail to do so in active problem solving.’
\citep{gomez:2005_species}
 
‘to date, adult primates’ failures on search tasks appear to exactly mirror the cases in which human toddlers perform poorly.’
\citep[p.\ 17]{santos:2009_object}
 
 
 
%--- end paste
%--------------- 
 
\footnotesize 
\bibliography{$HOME/endnote/phd_biblio}

\end{multicols}

\end{document}