%!TEX TS-program = xelatex
%!TEX encoding = UTF-8 Unicode

\documentclass[12pt]{extarticle}
% extarticle is like article but can handle 8pt, 9pt, 10pt, 11pt, 12pt, 14pt, 17pt, and 20pt text

\def \ititle {Origins of Mind}
 
\def \isubtitle {Lecture 08}
 
\def \iauthor {Stephen A. Butterfill}
\def \iemail{s.butterfill@warwick.ac.uk}
\date{}

%for strikethrough
\usepackage[normalem]{ulem}

\input{$HOME/Documents/submissions/preamble_steve_handout}

%\bibpunct{}{}{,}{s}{}{,}  %use superscript TICS style bib
%remove hanging indent for TICS style bib
%TODO doesnt work
\setlength{\bibhang}{0em}
%\setlength{\bibsep}{0.5em}


%itemize bullet should be dash
\renewcommand{\labelitemi}{$-$}

\begin{document}

\begin{multicols}{3}

\setlength\footnotesep{1em}


\bibliographystyle{newapa} %apalike

%\maketitle
%\tableofcontents




%--------------- 
%--- start paste


\def \ititle {Origins of Mind}
 
\def \isubtitle {Lecture 08}
 
 
 
\
 
 
 
\begin{center}
 
{\Large
 
\textbf{\ititle}: \isubtitle
 
}
 
 
 
\iemail %
 
\end{center}
 
 
 
\section{A Puzzle about Pointing}
 
‘infant pointing is best understood---on many levels and in many ways---as depending on uniquely human skills and motivations for cooperation and shared intentionality, which enable such things as joint intentions and joint attention in truly collaborative interactions with others (Bratman, 1992; Searle, 1995).’
\citep[p.\ 706]{Tomasello:2007fi}
 
\subsection{Why don’t ape’s point?}
 
‘there is not a single reliable observation, by any scientist anywhere, of one ape pointing for another’.
\citep[p.\ 507]{Tomasello:2010dy}
 
‘Although some apes, especially those with extensive human contact, sometimes point imperatively for humans […],
 
no apes point declaratively ever.’
\citep[p.\ 510]{Tomasello:2010dy}
 
‘to understand pointing, the subject needs to understand more than the individual goal-directed behaviour. She needs to understand that by pointing towards a location, the other attempts to communicate to her where a desired object is located; that the other tries to inform her about something that is relevant for her’
\citep[p.\ 6]{Moll:2007gu}.
 
‘the specific behavioral form — distinctive hand shape with extended index finger — actually emerges reliably in infants as young as 3 months of age (Hannan \& Fogel, 1987). […] why do infants not learn to use the extended index finger for these social functions at 3 – 6 months of age, but only at 12 months of age?’ \citep[p.\ 716]{Tomasello:2007fi}
 
\subsection{pointing vs linguistic communication}
 
‘the most fundamental aspects of language that make it such a uniquely powerful form of human cognition and communication---joint attention, reference via perspectives, reference to absent entities, cooperative motives to help and to share, and other embodiments of shared intentionality---are already present in the humble act of infant pointing.’ \citep[p.\ 719]{Tomasello:2007fi}
 
‘cooperative communication does not depend on language, […] language depends on it.’ \citep[p.\ 720]{Tomasello:2007fi}
 
‘Pointing may […] represent a key transition, both phylogenetically and ontogenetically, from nonlinguistic to linguistic forms of human communication.’ \citep[p.\ 720]{Tomasello:2007fi}
 
Comprehending pointing is not just a matter of locking onto the thing pointed to; it also involves some sensitivity to context \citep[see][]{Liebal:2010lr}.
 
\subsection{Pointing: referent and context}
 
‘Already by age 14 months, then, infants interpret communication cooperatively, from a shared rather than an egocentric perspective’ \citep[p.\ 269]{Liebal:2010lr}.
 
‘The fact that infants rely on shared experience even to interpret others’ nonverbal pointing gestures suggests that this ability is not specific to language but rather reflects a more general social-cognitive, pragmatic understanding of human cooperative communication’ \citep[p.\ 270]{Liebal:2010lr}.
 
 
 
\section{What is a communicative action?}
 
\subsection{A Gricean view}
 
First approximation: To communicate is to provide someone with evidence of an intention with the further intention of thereby fulfilling that intention
\citep[compare][chapter 14]{Grice:1989ha}.
 
The confederate means something in pointing at the left box if she intends:
 
\begin{enumerate}
 
\item
 
that you open the left box;
 
\item
 
that you recognize that she intends (1), that you open the left box; and
 
\item
 
that your recognition that she intends (1) will be among your reasons for opening the left box.
 
\end{enumerate}
 
(compare \citealp[p.\ 151]{Grice:1969pv}; \citealp[p.\ 544]{Neale:1992uw})
 
Theory of communicative action \citep[compare][]{Tomasello:2007fi}:
 
\begin{enumerate}
 
\item
 
Producing and understanding declarative pointing gestures constitutively involves embodying (?) shared intentionality.
 
\item
 
Embodying shared intentionality involves having knowledge about knowledge of your intentions about my intentions.
 
\end{enumerate}
 
\subsection{First alternative view}
 
‘No speaker needs to form any express intention … in order to mean by a word what it means in the language’
\citep[p.\ 473]{Dummett:1986mq}
 
‘Interpreting speech does not require making any inferences or having any beliefs about words, let alone about speaker intentions’
\citep[p.\ 62]{Millikan:1984ib}
 
\subsection{Davidsonian view}
 
‘meaning of whatever sort ultimately rests on intention’
\citep[p.\ 298]{Davidson:1992pl}
 
ulterior intentions: ‘intentions which lie as it were beyond the production of words … [such as] the intention of being elected mayor, of amusing a child, of warning a pilot of ice on the wings’ \citep[p.\ 298]{Davidson:1992pl}.
 
semantic intentions: intentions concerning the meaning of one’s utterance.
 
‘The intention to be taken to mean what one wants to be taken to mean is, it seems to me, so clearly the only aim that is common to all verbal behaviour that it is hard for me to see how anyone can deny it.’
\citep[p.\ 11]{Davidson:1994ol}
 
 
 
\section{Words and Communicative Actions}
 
 
 
\section{Syntax: Knowledge or Core Knowledge?}
 
‘It is of the essence of a belief [or knowledge] state that it be at the service of many distinct projects, and that its influence on any project be mediated by other beliefs.’
\citep[p.\ 337]{Evans:1981pc}
 
‘At the level of output, one who possesses the tacit knowledge that p is disposed to do and think some of the things which one who had the ordinary belief that p would be inclined to do an think (given the same desires).
At the level of input, one who possesses the state of tacit knowledge that p will very probably have acquired that state as the result of exposure to usage which supports of confirms … the proposition that p, and hence in circumstances which might well induce in a rational person the ordinary belief that p.’
\citep[p.\ 336]{Evans:1981pc}
 
‘(It is one of the fundamental differences between human thought and the information-processing that takes place in our brains that the Generality Constraint applies to the former but not to the latter. When we attribute to the brain computations whereby it localizes the sounds we hear, we ipso facto ascribe it to representations of the speed of sound and of the distance between the ears, without any commitment to the idea that it should be able to represent the speed of light or the distance between anything else.)’
\citep[p.\ 104, footnote 22]{Evans:1982je}
 
Modules ‘provide an automatic starting engine for encyclopaedic knowledge’
\citep[p.\ 194]{Leslie:1988ct}
 
‘The module … automatically provides a conceptual identification of its input for central thought … in exactly the right format for inferential processes’
\citep[p.\ 193--4]{Leslie:1988ct}
 


%--- end paste
%--------------- 
 
\footnotesize 
\bibliography{$HOME/endnote/phd_biblio}

\end{multicols}

\end{document}