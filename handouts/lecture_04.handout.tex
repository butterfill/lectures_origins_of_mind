%!TEX TS-program = xelatex
%!TEX encoding = UTF-8 Unicode

\documentclass[12pt]{extarticle}
% extarticle is like article but can handle 8pt, 9pt, 10pt, 11pt, 12pt, 14pt, 17pt, and 20pt text

\def \ititle {Origins of Mind}
 
\def \isubtitle {Lecture 02}
 
\def \iauthor {Stephen A. Butterfill}
\def \iemail{s.butterfill@warwick.ac.uk}
\date{}

%for strikethrough
\usepackage[normalem]{ulem}

\input{$HOME/Documents/submissions/preamble_steve_handout}

%\bibpunct{}{}{,}{s}{}{,}  %use superscript TICS style bib
%remove hanging indent for TICS style bib
%TODO doesnt work
\setlength{\bibhang}{0em}
%\setlength{\bibsep}{0.5em}


%itemize bullet should be dash
\renewcommand{\labelitemi}{$-$}

\begin{document}

\begin{multicols}{3}

\setlength\footnotesep{1em}


\bibliographystyle{newapa} %apalike

%\maketitle
%\tableofcontents




%--------------- 
%--- start paste



\def \ititle {Origins of Mind}
 
\def \isubtitle {Lecture 04}
 
 
 
\
 
 
 
\begin{center}
 
{\Large
 
\textbf{\ititle}: \isubtitle
 
}
 
 
 
\iemail %
 
\end{center}
 
 
 
\section{Recap: Three Questions}
 
Knowledge of objects depends on abilities to (i) segment objects, (ii) represent them as persisting and (iii) track their interactions.
 
\emph{Question 1} How do humans come to meet the three requirements on knowledge of objects?
 
\emph{Discovery 1} Infants manfiest all three abilities from around four months of age or earlier.
 
\emph{Discovery 2} Although abilities to segment objects, to represent them as persisting through occlusion and to track their causal interactions are conceptually distinct, they may all be consequences of a single mechanism (in humans and perhaps in other animals).
 
\emph{Question 2} What is the relation between the principles of object perception and infants’ looking behaviours?
 
The \emph{simple view} is the view that the principles of object perception are things that we know, and we generate expectations from these principles by a process of inference.
 
\emph{Discovery 3} The simple view generates systematically false predictions (about reaching).
 
\emph{Question 2a} Given that the simple view is wrong, what is the relation between the principles of object perception and infants’ competence in segmenting objects, object permanence and tracking causal interactions?
 
\emph{Question 2b} The principles of object perception result in ‘expectations’ in infants. What is the nature of these expectations?
 
\emph{Question 3} What is the relation between adults’ and infants’ abilities concerning physical objects and their causal interactions?
 
 
 
\section{The Parable of the Wrock}
 
 
 
\section{Core Knowledge and Modularity}
 
‘there are many separable systems of mental representations ... and thus many different kinds of knowledge. ... the task ... is to contribute to the enterprise of finding the distinct systems of mental representation and to understand their development and integration’
\citep[p.\ 1522]{Hood:2000bf}.
 
‘Just as humans are endowed with multiple, specialized perceptual systems, so we are endowed with multiple systems for representing and reasoning about entities of different kinds.’
\citep[p.\ 517]{Carey:1996hl}
 
‘core systems are largely innate, encapsulated, and unchanging, arising from phylogenetically old systems built upon the output of innate perceptual analyzers.’
\citep[p.\ 520]{Carey:1996hl}
 
A piece of \emph{core knowledge} is a representation in a core system.
 
The \emph{revised view}: the principles of object perception are not knowledge, but they are core knowledge.
 
\subsection{Objection}
 
‘there is a paucity of … data to suggest that they are the only or the best way of carving up the processing,
 
‘and it seems doubtful that the often long lists of correlated attributes should come as a package’
\citep[p.\ 759]{adolphs_conceptual_2010}
 
‘we wonder whether the dichotomous characteristics used to define the two-system models are … perfectly correlated …
[and] whether a hybrid system that combines characteristics from both systems could not be … viable’
\citep[p.\ 537]{keren_two_2009}
 
‘the process architecture of social cognition is still very much in need of a detailed theory’
\citep[p.\ 759]{adolphs_conceptual_2010}
 
‘In Fodor’s (1983) terms, visual tracking and preferential looking each may depend on modular mechanisms.’
\citep[p.\ 137]{spelke:1995_spatiotemporal}
 
\subsection{Modularity}
 
Fodor’s three claims about modules:
 
\begin{enumerate}
 
\item they are ‘the psychological systems whose operations present the world to thought’;
 
\item they ‘constitute a natural kind’; and
 
\item there is ‘a cluster of properties that they have in common’ \citep[p.\ 101]{Fodor:1983dg}.
 
\end{enumerate}
 
These properties include:
 
\begin{itemize}
 
\item domain specificity (modules deal with ‘eccentric’ bodies of knowledge)
 
\item limited accessibility (representations in modules are not usually inferentially integrated with knowledge)
 
\item information encapsulation (modules are unaffected by general knowledge or representations in other modules)
 
\item innateness (roughly, the information and operations of a module not straightforwardly consequences of learning; but see \citet{Samuels:2004ho}).
 
\end{itemize}
 
 
 
\section{Computation is the Real Essence of Core Knowledge}
 
‘modern philosophers … have no theory of thought to speak of. I do think this is appalling; how can you seriously hope for a good account of belief if you have no account of belief fixation?’
\citep[p.\ 147]{Fodor:1987rt}
 
‘Thinking is computation’
\citep[p.\ 9]{Fodor:1998ap}
 
The Computational Theory of Mind:
\begin{enumerate}
 
\item ‘Thoughts have their causal roles in virtue of, inter alia, their logical form.
 
\item ‘The logical form of a thought supervenes on the syntac¬tic form of the corresponding mental representation.
 
\item ‘Mental processes (including, paradigmatically, think¬ing) are computations, that is, they are operations defined on the syntax of mental representations, and they are reliably truth preserving in indefinitely many cases’
\citep[pp.\ 18--19]{Fodor:2000cj}
 
\end{enumerate}
 
‘the Computational Theory is probably true at most of only the mind’s modular parts. … a cognitive science that provides some insight into the part of the mind that isn’t modular may well have to be different, root and branch’
\citep[p.\ 99]{Fodor:2000cj}
 
Thinking isn't computation because:
 
\begin{enumerate}
 
\item Computational processes are not sensitive to context-dependent relations among representations.
 
\item Thinking sometimes involves being sensitive to context-dependent relations among representations as such.
 
\item Therefore, thinking isn’t computation \citep{Fodor:2000cj}.
 
\end{enumerate}
 
If a process is not sensitive to context-dependent relations, it will exhibit:

information encapsulation;
limited accessibility; and
domain specificity.
\citep{Butterfill:2007pe}
 
 
 
\section{Perception of Causation}
 
‘There are some cases … in which a causal impression arises, clear, genuine, and unmistakable, and the idea of cause can be derived from it by simple abstraction in just the same way as the idea of shape or movement can be derived from the perception of shape or movement’
\citep[p.\ 270--1]{Michotte:1946nz}
 
Infants seem also to distinguish launching from other sequences, much as adults do \citep{Leslie:1987nr}.
 
‘when there is a launching event beneath the overlap (or underlap event) timed such that the launch occurs at the point of maximum overlap, observers inaccurately report that the overlap is incomplete, suggesting that they see an illusory crescent.’
\citep[p.\ 461]{Scholl:2004dx}
 
Why does the illusory causal crescent appear? Scholl and Nakayama suggest a
 
‘a simple categorical explanation for the Causal Crescents illusion: the visual system, when led by other means to perceive an event as a causal collision, effectively ‘refuses’ to see the two objects as fully overlapped, because of an internalized constraint to the effect that such a spatial arrangement is not physically possible. As a result, a thin crescent of one object remains uncovered by the other one-as would in fact be the case in a straight-on billiard-ball collision where the motion occurs at an angle close to the line of sight.’
\citep[p.\ 466]{Scholl:2004dx}
 
‘just as the visual system works to recover the physical structure of the world by inferring properties such as 3-D shape, so too does it work to recover the causal … structure of the world by inferring properties such as causality’
\citep[p.\ 299]{Scholl:2000eq}
 
 
 
\section{Object Indexes and Causal Interactions}
 
The \emph{object-specific preview effect}: ‘observers can identify target letters that matched the preview letter from the same object faster than they can identify target letters that matched the preview letter from the other object.’
\citep[p.\ 2]{Krushke:1996ge}
 
‘objects are conceived: Humans come to know about an object’s unity, boundaries, and persistence in ways like those by which we come to know about its material composition or its market value.’
\citep[p.\ 198]{Spelke:1988xc}.
 
core knowledge of objects is a consequence of object indexes
\citep{Leslie:1998zk,Carey:2001ue}
 
 
 
 
 
 
%--- end paste
%--------------- 
 
\footnotesize 
\bibliography{$HOME/endnote/phd_biblio}

\end{multicols}

\end{document}