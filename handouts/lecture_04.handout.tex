%!TEX TS-program = xelatex
%!TEX encoding = UTF-8 Unicode

\documentclass[12pt]{extarticle}
% extarticle is like article but can handle 8pt, 9pt, 10pt, 11pt, 12pt, 14pt, 17pt, and 20pt text

\def \ititle {Origins of Mind}
 
\def \isubtitle {Lecture 02}
 
\def \iauthor {Stephen A. Butterfill}
\def \iemail{s.butterfill@warwick.ac.uk}
\date{}

%for strikethrough
\usepackage[normalem]{ulem}

\input{$HOME/Documents/submissions/preamble_steve_handout}

%\bibpunct{}{}{,}{s}{}{,}  %use superscript TICS style bib
%remove hanging indent for TICS style bib
%TODO doesnt work
\setlength{\bibhang}{0em}
%\setlength{\bibsep}{0.5em}


%itemize bullet should be dash
\renewcommand{\labelitemi}{$-$}

\begin{document}

\begin{multicols}{3}

\setlength\footnotesep{1em}


\bibliographystyle{newapa} %apalike

%\maketitle
%\tableofcontents




%--------------- 
%--- start paste



\def \ititle {Origins of Mind}
 
\def \isubtitle {Lecture 04}
 
 
 
\
 
 
 
\begin{center}
 
{\Large
 
\textbf{\ititle}: \isubtitle
 
}
 
 
 
\iemail %
 
\end{center}
 
 
 
\section{Recap: Three Questions}
 
Knowledge of objects depends on abilities to (i) segment objects, (ii) represent them as persisting and (iii) track their interactions.
 
\emph{Question 1} How do humans come to meet the three requirements on knowledge of objects?
 
\emph{Discovery 1} Infants manfiest all three abilities from around four months of age or earlier.
 
\emph{Discovery 2} Although abilities to segment objects, to represent them as persisting through occlusion and to track their causal interactions are conceptually distinct, they may all be consequences of a single mechanism (in humans and perhaps in other animals).
 
\emph{Question 2} What is the relation between the principles of object perception and infants’ looking behaviours?
 
The \emph{simple view} is the view that the principles of object perception are things that we know, and we generate expectations from these principles by a process of inference.
 
\emph{Discovery 3} The simple view generates systematically false predictions (about reaching).
 
\emph{Question 2a} Given that the simple view is wrong, what is the relation between the principles of object perception and infants’ competence in segmenting objects, object permanence and tracking causal interactions?
 
\emph{Question 2b} The principles of object perception result in ‘expectations’ in infants. What is the nature of these expectations?
 
\emph{Question 3} What is the relation between adults’ and infants’ abilities concerning physical objects and their causal interactions?
 
 
 
\section{Core Knowledge and Modularity}
 
‘Just as humans are endowed with multiple, specialized perceptual systems, so we are endowed with multiple systems for representing and reasoning about entities of different kinds.’
\citep[p.\ 517]{Carey:1996hl}
 
‘core systems are largely innate, encapsulated, and unchanging, arising from phylogenetically old systems built upon the output of innate perceptual analyzers.’
\citep[p.\ 520]{Carey:1996hl}
 
A piece of \emph{core knowledge} is a representation in a core system.
 
\subsection{Objection}
 
‘there is a paucity of … data to suggest that they are the only or the best way of carving up the processing,
 
‘and it seems doubtful that the often long lists of correlated attributes should come as a package’
\citep[p.\ 759]{adolphs_conceptual_2010}
 
‘we wonder whether the dichotomous characteristics used to define the two-system models are … perfectly correlated …
[and] whether a hybrid system that combines characteristics from both systems could not be … viable’
\citep[p.\ 537]{keren_two_2009}
 
‘the process architecture of social cognition is still very much in need of a detailed theory’
\citep[p.\ 759]{adolphs_conceptual_2010}
 
‘In Fodor’s (1983) terms, visual tracking and preferential looking each may depend on modular mechanisms.’
\citep[p.\ 137]{spelke:1995_spatiotemporal}
 
\subsection{Modularity}
 
Fodor’s three claims about modules:
 
\begin{enumerate}
 
\item they are ‘the psychological systems whose operations present the world to thought’;
 
\item they ‘constitute a natural kind’; and
 
\item there is ‘a cluster of properties that they have in common’ \citep[p.\ 101]{Fodor:1983dg}.
 
\end{enumerate}
 
These properties include:
 
\begin{itemize}
 
\item domain specificity (modules deal with ‘eccentric’ bodies of knowledge)
 
\item limited accessibility (representations in modules are not usually inferentially integrated with knowledge)
 
\item information encapsulation (modules are unaffected by general knowledge or representations in other modules)
 
\item innateness (roughly, the information and operations of a module not straightforwardly consequences of learning; but see \citet{Samuels:2004ho}).
 
\end{itemize}
 
 
 
%--- end paste
%--------------- 
 
\footnotesize 
\bibliography{$HOME/endnote/phd_biblio}

\end{multicols}

\end{document}