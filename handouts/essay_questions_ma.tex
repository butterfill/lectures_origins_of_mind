 %!TEX TS-program = xelatex
%!TEX encoding = UTF-8 Unicode

%\def \papersize {a5paper}
\def \papersize {a4paper}
%\def \papersize {letterpaper}

%\documentclass[14pt,\papersize]{extarticle}
\documentclass[12pt,\papersize]{extarticle}
% extarticle is like article but can handle 8pt, 9pt, 10pt, 11pt, 12pt, 14pt, 17pt, and 20pt text

\def \ititle {Origins of Mind}
\def \isubtitle {Seminar Tasks: MA/MPhil Version}
%comment some of the following out depending on whether anonymous
\def \iauthor {Stephen A.\ Butterfill}
\def \iemail{s.butterfill@warwick.ac.uk% \& corrado.sinigaglia@unimi.it
}
%\def \iauthor {}
%\def \iemail{}
%\date{}

%\input{$HOME/Documents/submissions/preamble_steve_paper3}
\input{$HOME/Documents/submissions/preamble_steve_report2}

%no indent, space between paragraphs
\usepackage{parskip}

%beautiful contents
\usepackage{tocloft}

%comment these out if not anonymous:
%\author{}
%\date{}

%for e reader version: small margins
% (remove all for paper!)
%\geometry{headsep=2em} %keep running header away from text
%\geometry{footskip=1.5cm} %keep page numbers away from text
%\geometry{top=1cm} %increase to 3.5 if use header
%\geometry{bottom=2cm} %increase to 3.5 if use header
%\geometry{left=1cm} %increase to 3.5 if use header
%\geometry{right=1cm} %increase to 3.5 if use header

% disables chapter, section and subsection numbering
\setcounter{secnumdepth}{-1} 

%avoid overhang
\tolerance=5000

%\setromanfont[Mapping=tex-text]{Sabon LT Std} 


%for putting citations into main text (for reading):
% use bibentry command
% nb this doesn’t work with mynewapa style; use apalike for \bibliographystyle
% nb2: use \nobibliography to introduce the readings 
\usepackage{bibentry}

%screws up word count for some reason:
%\bibliographystyle{$HOME/Documents/submissions/mynewapa} 
\bibliographystyle{apalike} 


\begin{document}

\nobibliography{$HOME/endnote/phd_biblio}


\setlength\footnotesep{1em}


\maketitle

%\renewcommand{\cftsecfont}{\normalsize}
%\setcounter{tocdepth}{1}
%\tableofcontents
%\clearpage

%
%\section{Guidelines}
%
%\subsection{Deadline}
%Essays are due 6pm on the Tuesday before your tutorial.
%
%I will not normally read late essays.
%


\section*{Warning}
This document may be updated during the course.
Tasks might change.
Please always check you have the latest version from \url{http://origins-of-mind.butterfill.com} before completing a task.
This version was last edited \today.

\section{Essay 1}

\subsection{Suggested question}

What is puzzling about  6-month-olds’ abilities to track physical objects’ causal interactions?  How might the puzzle be resolved?


\subsection{Peer review}
This essay will be subject to peer review. 
Another student in your seminar group will be assigned as your reviewer.
You should send the essay to your reviewer by 6pm two days before your seminar.



\subsection{Hint}
In this essay you might:
\begin{enumerate}
\item review some of the evidence that infants can track causal interactions (see readings, \citealp{spelke:1992_origins} or \citealp{Leslie:1987nr});
\item consider findings that are hard to reconcile with the claim that infants’ simply know that barriers stop objects (see \citealp{Hood:2000bf,Hood:2003yg})
\item attempt to resolve the conflict (potentially useful sources include \citealp{Haith:1998aq,Keen:2003xz})
\end{enumerate}

\subsection{Reading}

\bibentry{Spelke:1993no}

\bibentry{Hood:2000bf}



\subsection{Further reading}

\bibentry{spelke_1992:origins}

\bibentry{Leslie:1987nr}

\bibentry{Hood:2003yg}

\bibentry{santos:2006_cotton-top}

\bibentry{Haith:1998aq}

\bibentry{Keen:2003xz}

\bibentry{Fodor:1983dg}

\bibentry{Butterfill:2007pe}


\subsection{Where to find the readings}

All the readings are available online unless otherwise noted.

One fast way to find a paper is to copy its title into google scholar and search.  
To download the paper from the journal website, you may need to select ‘log in’ or ‘institutional log in’.

If you have trouble locating a resource, check the list of journals available here: \url{http://fs6jr8lx8q.search.serialssolutions.com/}


\subsection{Citations}
When citing articles in your essay, use the same system that my handouts and nearly all the readings use.
That is, put author-year in the main text (e.g.\ ‘Spelke et al have argued that ... (Spelke et al 1993, p.\ 22).’) and include the full citation in a list of references at the end.

A bibliography manager like Zotero can save you a lot of time.


\subsection{Length}
Your essay may not exceed 2500 words.
Essays longer than this words may be rejected without review.

Shorter is better, all things being equal.





\end{document}